\section{Quantum stack machine operation}\label{sec:stackmachineoperation}
This section describes the actual transitions of the stack
machine for each of the instructions in the machine.

\subsection{Machine transitions}\label{subsec:transitiondiagrams}
The majority of the transitions presented in this section are 
defined on 
a machine of type $\bms = (\cd,S,Q,D,N)$ as was 
introduced in \vref{subsec:basicmachinestate}.
As discussed in that section, labelling only affects the
transition of unitary transforms. All other instruction transitions 
ignore it, giving us:
\[ Ins(L(Q)) = L (Ins (Q)) \]
where $Ins$ is the transition of some other instruction.

The transition for the application of transformations will use \lbms, while
the transition for the add/remove control instructions 
uses the machine state of  \cms. 
The call instruction uses the 
state of the complete machine, \ms, which allows recursion.
All of these stages and their associated states were defined and discussed 
in  \vref{sec:qsmstate}.



\subsection{Node creation}\label{subsec:quantumstacknodecreation}
There are three instructions which allow us to create data on the 
stack and one which binds sub-nodes into a data type. These are 
\qsmins{QLoad, QCons, QMove} and \qsmins{QBind}. The transitions are shown
in \vref{fig:trans:nodeconstruction}.

The instructions do the following tasks:
\begin{description}
\item{\qsminswithp{QLoad}{nm\ \ket{i}}} --- Load a new \qubit{} named 
\qsminsparm{nm} on top of the quantum stack with the value \ket{i};
\item{\qsminswithp{QCons}{nm\ Cns}} -- Load a new datatype node on top of the
quantum stack with name \qsminsparm{nm} and value \qsminsparm{Cns}.
Sub-nodes are not bound  by this instruction.
\item{\qsminswithp{QMove}{nm}} --- Load a new classical node on top of the
quantum stack with name \qsminsparm{nm} and value taken from the
top of the classical stack. If the classical stack is empty, the
value is defaulted to $0$.
\item{\qsminswithp{QBind}{nm}} --- Binds a sub-node down the 
branch of the node to the datatype constructor on top of the 
quantum stack. Furthermore, the act of binding will
cause the newly bound sub-node to be renamed so that it is
hidden until an unbind is performed. \qsmins{QBind} 
uses the name supply,
$N$, to create the new  name for the sub-node. The machine will
generate an exception if the top of the quantum stack is not a
single branched datatype or if a node named \qsminsparm{nm} is not found. 
\end{description}


\begin{figure}[htbp]
\begin{tabular}{l}
$(\mathrm{QLoad}\ x\ \ket{k}{:}\cd,S,Q,D,N)  $ \\
$ \trspace\implies(\cd,S,x{:}[\ket{k}\to Q],D,N)$ \\
$(\mathrm{QCons}\ x\ c{:}\cd,S,Q,D,N) $ \\
$ \trspace\implies (\cd, S,x{:}[c\{\}\to Q],D,N) $ \\
$ (\mathrm{QMove}\ x{:}\cd,v{:}S,Q,D,N) $ \\
$ \trspace\implies (\cd,S,x{:}[\bar{v}\to Q],D,N)$ \\
$(\mathrm{QBind}\ z_0{:}\cd,S,x{:}[c\{z_1',\ldots,z_n'\}\to Q],D,N) $\\
$ \trspace \implies (\cd,S,x{:}[c\{z(N),z_1',\ldots,z_n'\}\to Q[z(N)/z_0]],D,N') $
\end{tabular}
\caption{Transitions for node construction}\label{fig:trans:nodeconstruction}
\end{figure}



\subsection{Node deletion}\label{subsec:nodedeletion}

Three different instructions, \qsmins{QDelete, QUnbind} and \qsmins{QDiscard}
 remove data from the 
quantum stack.  These instructions are the converses of \qsmins{QBind}, 
\qsmins{QLoad} and \qsmins{QMove}. Their transitions are shown in
\ref{fig:trans:nodedestruction} and \ref{fig:trans:noderemoval}.
The instructions do the following tasks:
\begin{description}
\item{\qsmins{QDelete}} --- removes the top node of the stack 
\emph{and any bound sub-nodes}. This instruction has no restrictions on
the number of sub-stacks or bindings in a data node;
\item{\qsmins{QDiscard}} --- removes the top node of the stack. In all cases,
the top node can only be removed when it has
a single sub-stack. For datatype nodes, \qsmins{QDiscard} also requires
there are no bound sub-nodes.
\item{\qsminswithp{QUnbind}{nm}} --- removes the
 first bound element from a data type 
\emph{provided it has a single sub branch}. The datatype node must be
 on top of the quantum stack. The newly unbound sub-node is renamed
 to \qsminsparm{nm}.
\end{description}


\begin{figure}[htbp]
\begin{tabular}{lll}
$(\mathrm{QDelete}{:}\cd,S,Q{:}[\ket{k_{ij}}\to Q_{ij}],D,N)$&$ \implies $&$(\cd,S,(Q_{00}+Q_{11}),D,N)$ \\[12pt]
$(\mathrm{QDelete}{:}\cd,S,DT{:}[c_i\{b_{ij}\} \to Q_i],D,N)$&$ \implies $&$(\cd,S,\sum_i(del(\{b_{ij}\}, Q_i)),D,N)$ \\[12pt]
$(\mathrm{QDelete}{:}\cd,S,I{:}[\overline{v}_i \to Q_i],D,N)$&$ \implies $&$(\cd,S,\sum_i Q_i,D,N)$ 
\end{tabular}
\caption{Transitions for destruction}\label{fig:trans:nodedestruction}
\end{figure}

For the \qsmins{QDelete} instruction, the type of node is irrelevant.
It will delete the node and, in the case of datatype nodes, any
bound nodes. This instruction is required to implement 
sub-routines that have parametrized datatypes as input arguments.
For example, the algorithm for determining the length of a
list is to return $0$ for the ``Nil'' constructor and 
add $1$ to the length of the tail list in the ``Cons'' constructor. 
When doing this, the elements of the list are deleted due to the 
linearity of \lqpl.
The compiler will have no way of determining the type of the 
elements in the list and
therefore could not generate the appropriate quantum split and discards.
The solution is to use a \qsmins{QDelete} instead.

The subroutine $del$ used in the transitions 
in \ref{fig:trans:nodedestruction}
will recursively rotate up and then delete the bound nodes of a datatype
 node.


\begin{figure}[htbp]
\begin{tabular}{lll}
$(\mathrm{QDiscard}{:}\cd,S,x{:}[\ket{k}\to Q],D,N) $&$ \implies$&$ (\cd,S,Q,D,N)$ \\[12pt]
$(\mathrm{QDiscard}{:}\cd,S,x{:}[c\{\} \to Q],D,N) $&$\implies $&$(\cd,S,Q,D,N)$ \\[12pt]
$(\mathrm{QDiscard}{:}\cd,S,x{:}[\overline{v} \to Q],D,N) $&$\implies$&$ (\cd,v{:}S,Q,D,N)$ \\[12pt]
\multicolumn{3}{l}{$(\mathrm{QUnbind}\ y{:}\cd,S,x{:}[c\{z_1',\ldots,z_n'\}\to Q],D,N) $} \\
\multicolumn{3}{r}{$\trspacefour\trspacethree\quad\ \ \implies\ \ \ (\cd,S,x{:}[c\{z_2',\ldots,z_n'\} \to Q[y/z_1']],D,N)$}
\end{tabular}
\caption{Transitions for removal and unbinding}\label{fig:trans:noderemoval}
\end{figure}



 The 
renaming is an integral part of the \qsmins{QUnbind}
instruction, as a compiler will not be able to know  the bound names of
a particular data type node. The instruction 
does \emph{not} delete the data type at the top of the stack or
the unbound node. If the top node is not a data type or has more than
a single branch or does not 
have any bound nodes, the machine will generate an exception. 

The machine ensures that it does not create  name capture issues
by rotating the bound node to the top of the sub-stack before
it does the rename. That is, given the situation as depicted in
the transitions, the quantum stack machine performs the 
following operations:

{\begin{singlespace}
\begin{enumerate}
\item{} $ Q' \leftarrow pull(z_1',Q)$;
\item{} $ Q'' \leftarrow  Q'[y/z_1'] $;
\item{} $z_1'$ is removed from the list of constructors;
\item{} The new quantum stack is now set to $x{:}[c\{z_2',\ldots,z_n'\} \to Q'']$.
\end{enumerate}
\end{singlespace}
}



\subsection{Stack manipulation}\label{subsec:stackmanipulation}
Most operations on a quantum stack affect only the top of the stack.
 Therefore,  the machine must have
 ways to move items up the stack. This requirement is met by the instructions
\qsmins{QPullup} and \qsmins{QName}. The transitions are shown in 
\vref{fig:trans:manipulation}.

The instructions do the following tasks:
\begin{description}
\item{\qsminswithp{QPullup}{nm}} --- brings the \emph{first} node named 
\qsminsparm{nm} to the top of the quantum stack. 
It is not an error to try pulling up a non-existent address. The original
stack will not be changed in that case.
\item{\qsminswithp{QName}{nm_1\ nm_2}} --- renames the first
node in the stack having \qsminsparm{nm_1} 
to \qsminsparm{nm_2}.
\end{description}

\begin{figure}[htbp]
\begin{tabular}{lll}
$(\mathrm{QPullup}\ x{:}\cd,S,Q,D,N) $&$\implies$&$ (\cd,S,\mathrm{\textsf{pull}}(x,Q),D,N)$ \\[12pt]
$(\mathrm{QName}\ x\ y{:}\cd,S,Q,D,N) $&$\implies$&$ (\cd,S,Q[y/x],D,N)$ 
\end{tabular}
\caption{Transitions for quantum stack manipulation}\label{fig:trans:manipulation}
\end{figure}

A \qsminswithp{QPullup}{nm} has the potential to be
an expensive operation as the node \qsminsparm{nm} may be 
deep in the quantum stack . In practice, many
 pullups interact  with only the top two or three elements of the
quantum stack.

The algorithm for pullup is based on preserving the bag of 
\emph{path signatures}.  A {path signature} for a node
consists of a bag of ordered pairs (consisting of the node name and
the branch constructor) where
every node from the top to the leaf is represented. 
Pulling up a
node will reorder the sub-branches below nodes to keep this invariant.

Due to the way  arguments of recursive subroutines are handled in \lqpl,
it is actually possible to get multiple nodes with the same name, however, this
does not cause a referencing problem as only the highest such node is actually
available in the \lqpl{} program.

\subsection{Measurement and choice}\label{subsec:measurementandchoice}
The instructions
\qsmins{Split}, \qsmins{Measure} and \qsmins{Use} start the task of 
operating on a node's partial stacks, while the fourth, \qsmins{SwapD} 
is used to iterate through the partial stacks. The transitions are shown in 
\vref{fig:trans:measures}.

The instructions do the following tasks:
\begin{description}
\item{\qsminswithp{Use}{(Lbl,EndLbl)}} --- uses the classical node at the top
of the quantum stack and executes the code at \qsminsparm{Lbl} for
each of its values. When done, jump to \qsminsparm{EndLbl}.
\item{\qsminswithp{Split}{([(c_1,lbl_1),\dots,(c_n,lbl_n)],EndLbl)}} --- uses the
datatype node at the top of the stack and
execute a jump to the code 
at \qsminsparm{lbl_i} when there is a branch having constructor
\qsminsparm{c_i}. Any constructors not mentioned in the instruction 
are removed from the node first. There is no ordering 
requirement on the pairs of constructors and labels in \qsmins{Split}. When all 
pairs have been processed, jump to \qsminsparm{EndLbl}.
\item{\qsminswithp{Measure}{(Lbl_{00},Lbl_{11},EndLbl)}} --- using the
\qubit{} node on top of the quantum stack, executes
 the code at the first two labels for 
 the \qsminsparm{00} and \qsminsparm{11} branches. 
The off-diagonal elements  of the \qubit{} will be
discarded. This implements a 
non-destructive measure of the \qubit. When done, jump to \qsminsparm{EndLbl}.
\item{\qsmins{SwapD}} --- signals the end of processing of dependent 
instructions and begins processing the next partial stack.
 When all values are processed, merges the results and returns
to the instruction labled by \qsminsparm{EndLbl} from the corresponding \qsmins{Measure}, \qsmins{Use}
or \qsmins{Split} instruction.
\end{description}
Note that in all cases above, these instructions execute a jump and the code following the 
instruction is not used per se. Naturally, the code following may be addressed by a label and therefore
will be executed. The \texttt{lqpl} compiler normally generates the code from one of these in
this layout:
\begin{verbatim}
      Enscope
      Split @c lbl3 (#Cons1,lbl1) (#Cons2, lbl2)
lbl1  Discard @c
      ...
      SwapD
lbl2  Discard @c
      ...
      SwapD
lbl3  Descope
      ...
\end{verbatim}
\begin{figure}[htbp]
\begin{tabular}{l}
$(\mathrm{Use}\ (\lbl_U,\lbl_{E}){:}\_,S,x{:}[\bar{v_i}\to Q_i],D,N) $ \\
$\trspacetwo\implies (\cd_U,S,x_1{:}v_1\to Q_1, 
	\dmpelemqc{}(S,[(x_i{:}v_i\to Q_i,\lbl_U)]_{i=2,\ldots,n},\lbl_E, 0){:}D,N)$ \\[12pt]
$(\mathrm{Split}\ ([(c_i,\lbl_i)],\lbl_E){:}\_,S,x{:}[c_i\{V_i\}\to Q_i],D,N) $ \\
$\trspacetwo\implies (\cd_1,S,x_1{:}c_1\{V_1\}\to Q_1, 
	\dmpelemqc{}(S,[(x_i{:}c_i\{V_i\}\to Q_i, \lbl_i)]_{i=2,\ldots,n},\lbl_E, 0){:}D,N)$ \\[12pt]
$(\mathrm{Meas}\ (\lbl_0, \lbl_1,\lbl_E){:}\_,S,x{:}[\ket{0}\to Q_0,\ket{1}\to Q_1],D,N) $ \\
$\trspacetwo\implies (\cd_0,S,(x_0{:}\ket{0}\to Q_0, 
       \dmpelemqc{}(S,[(x_1{:}\ket{1}\to 
	Q_1,\lbl_1)],\lbl_E, 0){:}D,N)$ \\[12pt]
\\
$(\mathrm{SwapD},S',Q,\dmpelemqc{}(S,
	[(Q_i,\lbl_i)]_{i=j,\ldots,m},\cdptr, Q'){:}D,N) 
       $ \\
$\trspacetwo\implies (\lblcd_j,S,Q_j, \dmpelemqc{}
	(S,[(Q_i,\lbl_i)]_{i=j+1,\ldots,m},\cdptr,Q+Q'){:}D,N)$ \\[12pt]
$(\mathrm{SwapD},S',Q,\dmpelemqc{}(S,[],\cdptr, Q'){:}D,N) 
       $ \\
$\trspacetwo\implies (\cd,S,Q+Q', D,N)$\\[12pt]
Specifically, for $\mathrm{Use}$, we have\\
$(\mathrm{SwapD},S',Q,\dmpelemqc{}(S,
	[(x_i{:}v_i\to Q_i,\lbl_U)]_{i=j,\ldots,m},\cdptr, Q'){:}D,N) 
       $ \\
$\trspacetwo\implies (\lblcd_U,S,x_j{:}v_j\to Q_j, \dmpelemqc{}
	(S,[(x_i{:}v_i\to Q_i,\lbl_U)]_{i=j+1,\ldots,m},\cdptr,Q+Q'){:}D,N)$ \\[12pt]
\\
For $\mathrm{Split}$, we have\\
$(\mathrm{SwapD},S',Q,\dmpelemqc{}(S,[(x_i{:}c_i\{V_i\}\to Q_i, \lbl_i)]_{i=j,\ldots,m},\cdptr, Q'){:}D,N) 
       $ \\
$\trspacetwo\implies (\lblcd_j,S,x_j, \dmpelemqc{}(S,[(x_i{:}c_i\{V_i\}\to Q_i, \lbl_i)]_{i=j+1,\ldots,m},\cdptr,Q+Q'){:}D,N)$ \\[12pt]
Finally, for $\mathrm{Meas}$, 
$(\mathrm{SwapD},S',Q, \dmpelemqc{}(S,
	[(x_1{:}\ket{1}\to Q_1,\lbl_1)],\cdptr, Q'){:}D,N) 
       $ \\
$\trspacetwo\implies (\lblcd_1,S,x_1{:}\ket{1}\to Q_1,  
	\dmpelemqc{}(S,[],\cdptr,Q+Q'){:}D,N)$ \\[12pt]
\end{tabular}
\caption{Transitions for quantum node choices}\label{fig:trans:measures}
\end{figure}

Each of the code fragments pointed to 
by the instruction labels \emph{must} end with the instruction
\qsmins{SwapD}. The \qsmins{SwapD} instruction  will trigger execution
 of the code associated with the 
next partial stack. 


The \qsmins{QUnbind} is meant to be used at the start of the dependent
 code of a \qsmins{Split} instruction.
The  sequencing to process a datatype node is to
do a \qsmins{Split}, then in each of the dependent blocks, execute
\qsmins{QUnbind} instructions, possibly interspersed with 
\qsmins{QDelete} instructions when the bound node is not further used
in the code.
This is always concluded with a \qsmins{QDiscard} that discards the data node
which was the target of the \qsmins{Split}.

In the following discussion there are no significant differences between the
\qsmins{Split} and \qsmins{Measure}. The action of
\qsmins{Split} is described in detail.

The \qsmins{Split}, \qsmins{Measure} and \qsmins{Use}
 instructions make  use of the dump. The dump element
used by these instructions consists of four parts: 
\begin{itemize}
\item{}\emph{The return label}. This is used 
when the control group is complete.
\item{}\emph{The remaining partial stacks}. A list consisting 
of pairs of quantum stacks and 
their corresponding label. These partial stacks are the ones waiting
 to be processed by the control group. 
\item{}\emph{The result quantum stack}. This quantum stack 
 accumulates the merge result of 
processing each of the control groups partial stacks. This is initialized
to an empty stack with a zero trace.
\item{}\emph{The saved classical stack}. The classical stack is reset to this
value at the start of processing a partial stack and at the end. This 
occurs each time an \qsmins{SwapD} instruction is executed.
\end{itemize}


The instruction 
\qsminswithp{Split}{[(c1,l1),(c2,l2)]} begins with the creation
of a  dump entry holding
$[(c2 \rightarrow Q_2,l2)]$ as the
list of partial stacks and label pairs. The 
dump entry will hold \qsminsparm{0} quantum stack as
the result stack, 
the current state of the classical stack  and the address of the 
instruction following the \qsmins{Split}.
The final processing of the 
\qsmins{Split} instruction sets the current quantum stack to 
$c0$ and sets the next code 
to be executed to be at $l0$. 

When the \qsmins{SwapD} instruction is executed, the dump will
again be changed. First the current quantum stack will be merged with
the result stack on the dump. Then the next pair of 
partial quantum stack $P_q\ (=c2\rightarrow Q_2) $ 
and code pointer $l2$ is removed from the execution list. 
The  current quantum stack is set to $P_q$ and the code pointer 
is set to $l2$. Finally, the classical stack is reset to the one 
saved in the dump element.

When the partial stack  list on the dump element is empty, 
the \qsmins{SwapD} instruction
will merge the current quantum stack with the result stack and then
set the current quantum stack to that result. 
The classical stack is reset to the one
saved on the dump, the code pointer is set to the return location
 saved in the
dump element and the dump element is removed. Program execution then
continues from the saved return point.

Normally, the first few instructions pointed to by the \qsminsparm{Label} in 
the pairs of constructor and code labels will unbind any bound nodes and 
delete the node at the top of the stack. 
QSM does not \emph{require} this, hence, it is possible to implement both
destructive and non-destructive measurements and data type deconstruction.

\paragraph{Using classical values.} The \qsmins{Use} instruction introduced
above differs from the instructions
 \qsmins{Split} and \qsmins{Measure} in that it works
on a node that may an unbounded number of sub-nodes.
The \qsminswithp{Use}{lbl} instruction moves 
all the partial stacks to the quantum stack, one at a time, and then
executes the code at \qsminsparm{lbl}
 for the resulting machine states. Normally, this 
code will start with a \qsmins{QDiscard}, which will put the node
value for that partial stack onto the
classical stack, and finish with an  \qsmins{SwapD} to
 trigger the processing of the next partial stack.

The dump and \qsminsparm{SwapD} processing for a \qsminswithp{Use}{lbl}
 is the same
as for a \qsmins{Split} or \qsmins{Measure}. The execution list pairs
will all have the same label, the \qsminsparm{lbl} on the instruction.

\subsection{Classical control}\label{subsec:classicalcontrol}
The machine provides the three instructions \qsmins{Jump, CondJump} and
\qsmins{NoOp} for branch control. Jumps are allowed only in a
forward direction. The transitions for these are shown in 
\vref{fig:trans:classicalcontrol}. The instructions do the following tasks:
\begin{description}
\item{\qsminswithp{Jump}{lbl}} --- causes execution to continue with 
the code at \qsminsparm{lbl}.
\item{\qsminswithp{CondJump}{lbl}} --- examines the top of the classical stack. 
When it is \qsmfalse, execution will continue with 
the code at \qsminsparm{lbl}. If it is any other value, execution continues with
the instruction following the \qsmins{CondJump}.
\item{\qsmins{NoOp}} --- does nothing in the machine. Execution continues with 
the instruction following the \qsmins{NoOp}.
\end{description}

\begin{figure}[htbp]

\begin{tabular}{lll}
$(\mathrm{Jump}\ \lbl_J{:}\cd,S,Q,D,N) $&$\implies $&$ (\lblcd_J,S,Q,D,N)$ \\[12pt]
$(\mathrm{CondJump}\ \lbl_J{:}\cd, \qsmfalse{}{:}S,Q,D,N) $ &$\implies $&$ (\lblcd_J,S,Q,D,N)$ \\[12pt]
$(\mathrm{CondJump}\ \lbl_J{:}\cd, \qsmtrue{:}S,Q,D,N) $ &$\implies $&$ (\cd,S,Q,D,N)$ \\[12pt]
$(\mathrm{NoOp}{:}\cd, S,Q,D,N) $ &$\implies $&$ (\cd,S,Q,D,N)$ 
\end{tabular}
\caption{Transitions for classical control.}\label{fig:trans:classicalcontrol}
\end{figure}


 No changes are made to the classical stack,
the quantum stack or the dump by these instructions.
While \qsmins{NoOp} does nothing, it is allowed as the target of a jump. 
This is 
used by the \lqpl{} compiler in the  code generation 
 as the instruction following a \qsmins{Call}.


\subsection{Operations on the classical stack}\label{subsec:operationsonclassicalstack}
The machine has five instructions that affect the classical stack directly.
They are \qsmins{CGet, CPut, CPop, CApply} and \qsmins{CLoad}, with 
transitions shown in \ref{fig:trans:classicalops}. The instructions perform
the following tasks:
\begin{description}
\item{\qsmins{CPop}} --- destructively removes the top element of the
classical stack.
\item{\qsminswithp{CGet}{n}} --- copies the $n^{\text{th}}$ element of the 
classical stack to the top of the classical stack.
\item{\qsminswithp{CApply}{\mathrm{\textsf{op}}}} --- applies the operation \qsminsparm{op} to 
the top elements of the classical stack, replacing them with the result of 
the operation. Typically, the \qsminsparm{\mathrm{\textsf{op}}}
 is  a binary operation such as 
\emph{add}.
\item{\qsminswithp{CLoad}{v}} --- places the constant  \qsminsparm{v} on
 top  of the classical stack.
\end{description}

\begin{figure}[htbp]
\begin{tabular}{lll}
$(\mathrm{CPop} {:}\cd,v{:}S,Q,D,N) $ &$\implies  $ &$(\cd,S,Q,D,N)$ \\[12pt]
$(\mathrm{CGet}\ n {:}\cd,v_1{:}\cdots{:}v_n{:}S,Q,D,N)  $ &$\implies $ &$ (c,v_n{:}v_1{:}\cdots{:}v_n{:}S,Q,D,N)$ \\[12pt]
$(\mathrm{CPut}\ n {:}\cd,v_1{:}\cdots{:}v_n{:}S,Q,D,N)  $ &$\implies  $ &$(c,v_1{:}\cdots{:}v_1{:}S,Q,D,N)$ \\[12pt]
$(\mathrm{CApply}\ \mathrm{\textsf{op}}_n{:}\cd,v_1{:}\cdots{:}v_n{:}S,Q,D,N)  $ &$\implies  $ &$(\cd,\mathrm{\textsf{op}}_n(v_1,\ldots,v_n){:}S,Q,D,N)$ \\[12pt]
$(\mathrm{CLoad}\ n\ {:}\cd,S,Q,D,N) $ &$ \implies $ &$ (\cd,n{:}S,Q,D,N)$
\end{tabular}
\caption{Transitions for classical stack operations.}\label{fig:trans:classicalops}
\end{figure}

\subsection{Unitary transformations and quantum control}\label{subsec:trans:unitarytransformations}
The QS-Machine has three instructions which add or remove \qubits{} (and other nodes)
from quantum control. The instruction transitions in this group
are defined directly on \cms{} or \lbms,  
as they will either affect the control
stack (\qsmins{AddCtrl, QCtrl, UnCtrl}) or need to take into account
 the labelling of the quantum stacks (\qsmins{QApply}). 

The first three instructions do not affect the actual state
of the quantum stack, classical stack or dump.
The \qsmins{QApply} does affect the state of the quantum stack.
The transitions  are
shown in \vref{fig:trans:unitarytransform}.

 The instructions perform
the following tasks:
\begin{description}
\item{\qsmins{AddCtrl}} --- starts a new control point on the control stack.
\item{\qsmins{QCtrl}} --- adds the node at the top of the 
quantum stack, together
with any dependent sub-nodes to the control stack.
\item{\qsmins{UnCtrl}} --- removes \emph{all} the nodes in the top control
point of the control stack.
\item{\qsminswithp{QApply}{n\ T}} --- parametrized the transform 
\qsminsparm{T} with the top $n$ elements of the classical stack and
then applies the parametrized transform to quantum stack. Control is respected
because of the labelling of the quantum stack.
\end{description}

\begin{figure}[htbp]
\begin{tabular}{l}
$(\mathrm{AddCtrl}{:}\cd,C,[(S_i,L(Q_i),D_i,N_i)]_{i=1,\cdots n})]) $ \\
$\trspace\qquad\implies\ \quad (\cd,id{:}C,[(S_i,L(Q_i),D_i,N_i)]_{i=1,\cdots n})])$\\[12pt]

$(\mathrm{QCtrl}{:}\cd,f{:}C,[(S_i,L(Q_i),D_i,N_i)]_{i=1,\cdots n})]) $ \\
$\trspace\qquad\implies\ \quad (\cd,(g\circ f){:}C,[(S^{'}_j,L(Q_j)^{'},D^{'}_j)]_{j=1,\cdots m})])$\\[12pt] 
$(\mathrm{UnCtrl}{:}\cd, f{:}C,[(S_i,L(Q_i),D_i,N_i)]_{i=1,\cdots n})]) $ \\
$\trspace\qquad\implies\ \quad (\cd,C,[(S^{''}_j,L(Q_j)^{''},D^{''}_j)]_{j=1,\cdots p})])$\\[12pt]
$(\mathrm{QApply}\ m\ t{:}\cd, (v_1{:}\cdots {:}v_m{:}S),L(Q),D,N)  $ \\
$\trspace\qquad\implies\ \quad (\cd,S,
           \mathrm{\textsf{cTrans}}([v_1,\ldots,v_m],t,L(Q)),D,N)$ 
\end{tabular}
\caption{Transitions for unitary transforms}\label{fig:trans:unitarytransform}
\end{figure}

The function $cTrans$ in the transition for \qsmins{QApply} must first 
create the transform. In most cases, this is a fixed  transform (e.g., \nottr,
\Had), but both \inlqpl{rotate} and the \inlqpl{UM}
 transforms are parametrized. 
The transform  \inlqpl{rotate} is used in the quantum Fourier transform and 
\inlqpl{UM} is the $a^x \mod N$ transform used in order finding.

When the top node is a \qubit, the function expects its required number
of \qubits{} to be the top nodes. For example, a \Had{} expects only $1$, a 
$swap$ expects 2 and an \inlqpl{UM}
 will expect as many \qubits{} as $N$ requires
\bit{}s.

When a transform is applied to a  datatype node the
 machine will attempt to rotate up the 
required number of \qubits{} to the top, perform the operation and 
then re-rotate the datatype node back to the top. 

The first step is to rotate the bound 
nodes of the datatype node starting at the 
 left and proceeding to the  right. Left to right is determined by the 
 ordering in the original constructor expression used to create
the datatype node. The machine will throw 
an exception if  there are insufficient bound nodes (e.g., 
\inlqpl{Nil} for a list) or if the rotation would be indeterminate. 
The machine considers a rotation for a transform to be 
indeterminate whenever the subject datatype node has more than one
sub-stack. For example, this means a transform can not be applied to
an \inlqpl{Either} that is a mix of  \inlqpl{Left(|0>)} and
\inlqpl{Right(|1>)}. 

When the rotation of the \qubits{} succeeds
 the function will transform the top parts of
the stack into a matrix $Q$ of appropriate size ($2\times2$ for a 
$1-$\qubit{} transform, $16\times16$ for 
a $4-$\qubit{} and so forth) with entries in the matrix
being the sub-stacks of the \qubits{} used in the transform. 

At this point, the control labelling 
of the quantum stack is considered and one of
the following four transforms will happen. If the actual transform
is named $T$, the result will be:
\begin{equation*}
\mathrm{cTrans}\ T\ L(Q) =
\begin{cases}
L(Q) & L= \mathrm{\texttt{IdOnly}}\\
L(T Q) & L= \mathrm{\texttt{Left}}\\
L(Q T^{*}) & L= \mathrm{\texttt{Right}}\\
L(T Q T^{*}) & L= \mathrm{\texttt{Full}}
\end{cases}
\end{equation*}

Following this the quantum stack is reformed from the resulting 
matrix. 

\subsection{Function calling and returning}\label{subsec:functioncalling}
The \qsmins{Call} and \qsmins{Return} instructions are used for 
function calling.
The \qsmins{Call} instruction is the only instruction that needs to 
directly work on \ms, the infinite list of \cms{} items. The transition
for this is defined in terms of a subordinate function \emph{enterF}
 which is 
defined on \bms. Its transition is also described below.

Recall the QS-machine stages have the states:
\begin{gather*}
\bms = (\cd,S,Q,D,N)\\
\cms = (\cd, C , [(S,L(Q),D,N)])\\
\ms = \Inflist{(\cd, C , [(S,L(Q),D,N)])} 
\end{gather*}
For the state \ms{}, an infinite list will be expressed as
\[H_0 \ilsep T = H_0 \ilsep H_1 \ilsep H_2 \ilsep \cdots\]
where $H$ is an element of the correct type for the infinite 
list.
 
 The instructions do the following tasks:
\begin{description}
\item{\qsminswithp{Call}{n\ lbl}} --- Calls the subroutine at \qsminsparm{lbl}, 
copying the top \qsminsparm{n} elements of the classical stack to a 
classical stack for the subroutine. 
\item{\qsminswithp{Return}{n}} --- Uses the return label in the
head of the dump to return from the subroutine. It also copies
 the top \qsminsparm{n} elements from the classical stack and
places them on top of the saved classical stack from the dump element.
\end{description}

\begin{figure}[htbp]
\begin{tabular}{l}
$ (\mathrm{Call}\ n\ \lbl_C{:}\cd,C, [(S_i,L(Q)_i,D_i,N_i)])_0 \ilsep T$\\
$\trspace\implies (\cd,C, [(S_i,L(\emptyset)_i,D_i,N_i)])_0 \ilsep
{lift\ (enterf\ n\ \lbl_C)\ T}$\\
$enterf\ n\ \lbl_C (\cd,v_1{:}\cdots{:}v_n{:}S,Q,D,N) $ \\
$\trspace\implies (\lblcd_C,[v_1,\ldots,v_n],Q,R(S,\cdptr){:}D,N)$ \\
$(\mathrm{Return}\ n,v_1{:}\cdots{:}v_n{:}S',Q,R(S,\cdptr){:}D,N) $ \\
$\trspace\implies (\cd,[v_1,\ldots,v_n]{:}S,Q,D,N)$ \\
\end{tabular}
\caption{Transitions for function calls.}\label{fig:trans:functioncalls}
\end{figure}

To illustrate how \qsmins{Call} is being processed, consider
the following diagram:
\begin{align*}
(0)&&M_0 &\ilsep &M_1 &\ilsep &M_2 &\ilsep &M_3 &\ilsep \dots \\
(1)&\ (\qsmins{Call}\ f):\quad&0 &\ilsep &f \cdot M_1 &\ilsep &f \cdot M_2 &\ilsep &f \cdot M_3 &\ilsep \dots \\
(2)&\ (\qsmins{Call}\ f):\quad&0 &\ilsep &0 &\ilsep &f \cdot f \cdot M_2 &\ilsep &f \cdot f \cdot M_3 &\ilsep \dots \\
(3)&\ (\qsmins{Call}\ f):\quad&0 &\ilsep &0 &\ilsep &0 &\ilsep &f \cdot f \cdot f \cdot M_3 &\ilsep \dots \\
&&&&&\vdots
\end{align*}

At the start, in line (0), the machine has state $M_0\ilsep M_i$. After the
first call to $f$, at line (1), 
the head of the infinite list state has been zeroed out, 
indicating divergence. However, at every position further down the
infinite list, the subroutine $f$ is entered.

Continuing to line (2) and calling $f$ again, the divergence has moved 
one position to the right and we now have a state of
 $0\ilsep 0\ilsep f \cdot f \cdot M_i$. Line (3) follows the same pattern.
 Thus, the further 
along in the infinite list one goes, the greater the \emph{call depth}.

% For  details of how this is
%handled in the Haskell implementation of the quantum stack machine, 
%see \vref{subsubsec:QSM:recursivefunctiontransitions}.

The \qsmins{Call} and \qsmins{Return} instructions use a dump element
as part of subroutine linkage.
The \qsminswithp{Call}{i\ lbl} instruction creates a dump element to store 
the 
current classical stack and the address of the instruction following
the \qsmins{Call} instruction. 
The \qsminswithp{Return}{n} instruction will use the top dump element
to reset the code pointer  to the saved return location.
\qsmins{Return} also takes the classical stack from the top dump element and the
top \qsminsparm{n} values from the current classical stack are added
to the top of it. \qsmins{Return} then removes the top dump element.

