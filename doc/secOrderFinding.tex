\section{Order finding}\label{sec:orderfinding}
Shor's algorithm for factoring depends on the quantum algorithm for 
order finding. Shor's algorithm may be summarized
by the following three steps:
\begin{itemize}
\item{}Setup;
\item{}Call order finding (a quantum algorithm);
\item{}Check to see if an answer has been found, repeat if not.
\end{itemize}

In this section, we concentrate on the quantum portion, order finding. 
Theoretical details may be found in \cite{neilsen2000:QuantumComputationAndInfo}.

The code for the main order finding function, \inlqpl{OrderFind} 
is shown in \ref{fig:orderfind}

\begin{figure}[htbp]
\lstinputlisting[style=linqpl]{examplecode/orderfind/OrderFind.qpl}
\caption{\lqpl{} function for order finding.}\label{fig:orderfind}
\end{figure}

Beginning at lines 1 and 2, the program imports files which
provide various other functions. At line \ref{of:signature}, the 
function is declared to take two classical integers as input and return
one probabalistic integer. The first two lines of code, \ref{of:sizeofn} and
\ref{of:sizeoft} compute the number of \qubits{} required for the algorithm
to work with a maximum error of $\frac{1}{4}$. Increasing 
\inlqpl{sizeT} will reduce the error.

Lines \ref{of:makelist} through \ref{of:makeqmod} complete the initial setup
for the algorithm. These include: Creating \inlqpl{tList}, 
a list of \inlqpl{sizeT} \qubits{} initialized to zero; applying the Hadamard 
transform to each \qubit; creating \inlqpl{xQm}, a \inlqpl{QuintMod} with 
the value of the input parameter \inlqpl{x}. See \vref{subsec:examplemodarith}
for the details of modular arithmetic.

Line \ref{of:dotransform} performs the modular exponentiation and is shown in
\ref{fig:modularpowerof}. This is similar to the exponentiation function in 
\vref{fig:quantummodularexp} and has just been modified to work with an 
exponent in list form and to retain the exponent.

\begin{figure}[htbp]
\lstinputlisting[style=linqpl]{examplecode/orderfind/powerFind.qpl}
\caption{\lqpl{} code for modular power by a list}\label{fig:modularpowerof}
\end{figure}

On completion of the exponentiation, the actual result is discarded as
it has no further effect on the algorithm.  Then, the inverse quantum Fourier
transform (shown in \ref{fig:inverseqft}
and \ref{fig:inverserotate}) is applied to \inlqpl{tList} and it is measured to produce
a probabalistic integer in line \ref{of:measure}.

\begin{figure}[htbp]
\lstinputlisting[style=linqpl]{examplecode/orderfind/inverseQft.qpl}
\caption{Function to apply the inverse quantum Fourier transform}
\label{fig:inverseqft}
\end{figure}


\begin{figure}[htbp]
\lstinputlisting[style=linqpl]{examplecode/orderfind/inverseRotate.qpl}
\caption{Function to apply inverse rotations as part of the inverse QFT}
\label{fig:inverserotate}
\end{figure}

The final part of the algorithm is normally described as using a continued fraction
algorithm to compute the potential order. This is the same as dividing 
$2^\text{\inlqpl{sizeT}}$ by the greatest common division of itself and the result.

At this point, since this is a probabilistic algorithm, the calling function would need to 
check whether the result is correct. If so, it may then be used in the completion of the 
factoring algorithm.

Various other support functions for this algorithm are shown in \ref{fig:ofutils}
and  \ref{fig:oflistutils}.

\begin{figure}[htbp]
\lstinputlisting[style=linqpl]{examplecode/orderfind/Utils.qpl}
\caption{GCD and import of other ulitlties}
\label{fig:ofutils}
\end{figure}


\begin{figure}[htbp]
\lstinputlisting[style=linqpl]{examplecode/orderfind/ListUtils.qpl}
\caption{Various list ulitlties}
\label{fig:oflistutils}
\end{figure}


