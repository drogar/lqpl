\section{Instructions}\label{subsec:repauxinstructions}
Creation of reasonable set of instructions, balancing brevity and usefulness
has been an interesting task. The list of instructions and brief
descriptions of them are presented in \vref{tab:instructionlist}. The 
transitions of these are presented formally 
in \vref{sec:stackmachineoperation}.

{\begin{singlespace}
\tablecaption{QSM instruction list}\label{tab:instructionlist}
\tablehead{\hline \textbf{Instruction}&\textbf{Arguments}&\textbf{Description}\\ \hline}
\tabletail{\hline \multicolumn{3}{r}{\emph{Continued on next page}}\\}
\tablelasttail{\hline}
\begin{supertabular}{|p{.85in}|p{1.2in}|p{3.69in}|}
\qsmins{QLoad}&\emph{nm:Name}, \emph{k::\qubit}&Creates \qubit{} named \emph{nm}
 and sets the value to \ket{k}.\\ & & \\
\qsmins{QMove}&\emph{nm:Name} &Creates an integer or Boolean named \emph{nm}
 and sets its value to the top of the classical stack.\\ & & \\
\qsmins{QCons}&\emph{nm:Name}, \emph{c::constructor} &Creates a data type element
with the value \emph{c}. Note that if the constructor requires sub-elements,
this will need to be followed by \qsmins{QBind} instructions.\\ & & \\
\qsmins{QBind}&\emph{nm:Name}&Binds the node $[nm]$ to 
the data element currently on 
the top of the stack.\\
\hline
\qsmins{QDelete}&$\emptyset$&Deletes the top node and any
 bound nodes in the quantum stack.\\ & & \\
\qsmins{QDiscard}&$\emptyset$&Discards the node on top of
 the quantum stack.\\ & & \\
\qsmins{QUnbind}&\emph{nm:Name}&Unbinds the first bound node 
from the data element at
the top of the stack and assigns it as \emph{nm}.\\ 
\hline
\qsmins{QPullup}&\emph{nm::name}&Pulls the node named 
\emph{nm} to the top of the 
quantum stack.\\ & & \\
\qsmins{QName}&\emph{nm1::name}, \emph{nm2::name}&Renames 
the node named \emph{nm1} to 
\emph{nm2}.\\
\hline
\qsmins{AddCtrl}&$\emptyset$&Marks  the start
of a control point in the control stack. Any following \qsmins{QCtrl} 
instructions will add the top node to this control point.\\ & & \\
\qsmins{QCtrl}&$\emptyset$&Moves the top element of the quantum stack
to the control stack. Recursively moves any bound nodes to the control
stack when the element is 
a constructed data type.\\ & & \\
\qsmins{UnCtrl}&$\emptyset$&Moves all items in the  control stack at the
current control point back to the quantum stack.\\ & & \\
\qsmins{QApply}&\emph{i::Int},\emph{t:Transform}&Parametrizes the transform \emph{T} with
the top \emph{i} elements of the classical stack and applies it to the
 quantum stack.\\
\hline
\qsmins{Measure}&\emph{l0::Label}, \emph{l1::Label}&Measures 
the \qubit{} on top of the
quantum stack and sets up the dump for execution of
 the code at \emph{l0} for the $00$ sub-branch and 
the code at \emph{l1} for the $11$ sub-branch.\\ & & \\
\qsmins{Split}&\emph{cls::[(constructor, Label)]}&Splits the data node at 
the top of the
quantum stack and sets up the dump  for execution of the code at 
the $i$-th label for the $i$-th  sub-branch.\\ & & \\
\qsmins{Use}&\emph{lbl::Label}&Uses the classical 
(integer or Boolean) node on top of the
quantum stack and sets up the dump to for execution of the code at 
\emph{lbl} for each of the sub-branches.\\ & & \\
\qsmins{SwapD}&$\emptyset$&Merges the current quantum stack
 with the results stack of the
dump, activates the next partial stack to be processed
 and jumps to the code at the corresponding 
label. When there are no more partial stacks, the instruction 
merges the current
stack with the the results stack and sets that as the new quantum stack.\\
\hline
\qsmins{Call}&\emph{i::Int}, \emph{ep::EntryPoint}&For the first element of the 
infinite list of states, sets the values at the leaves of the 
quantum stack to $0$. 
For the remainder of the list, the instruction jumps to the 
subroutine at \emph{ep}, saving
the return location and classical stack on the dump. It copies the
top \emph{i} elements of the classical stack for the subroutine.\\ & & \\
\qsmins{Return}&\emph{i::Int}&Restores the location and classical stack from the
dump, copies the top \emph{i} items of the current classical stack to the
top of the restored classical stack.\\
\hline
\qsmins{Jump}&\emph{lbl::Label}&Jumps \emph{forward} to the label \emph{lbl}.\\ & & \\
\qsmins{CondJump}&\emph{lbl::Label}&If the top of the classical stack is the
value \terminalio{false}, jumps \emph{forward} to the label \emph{lbl}.\\ & & \\
\qsmins{NoOp}&$\emptyset$&Does nothing.\\
\hline
\qsmins{CGet}&\emph{i::Int}&Copies the \emph{i}-th element of the 
classical stack to the
top of the classical stack. A negative value for \emph{i} indicates the
instruction should copy the $|i|^{\text{th}}$ value from the
bottom of the classical stack.\\ & & \\
\qsmins{CPut}&\emph{i::Int}&Copies the top of the classical stack to the 
\emph{i}-th element of the classical stack.  A negative value for 
\emph{i} indicates the
instruction should place the value into the $|i|^{\text{th}}$ location
 from the
bottom of the classical stack.\\ & & \\
\qsmins{CPop}&$\emptyset$&Pops off (and discards) the top element of 
the classical stack.\\ & & \\
\qsmins{CLoad}&\emph{v::Either Int Bool}&Pushes \emph{v} onto the classical stack.\\ & & \\
\qsmins{CApply}&\emph{op::Classical Op}&Applies \emph{op} to the top elements of
the classical stack, replacing them with the result of the operation.\\
\end{supertabular}
\end{singlespace}
}
