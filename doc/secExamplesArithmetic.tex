\section{Quantum arithmetic}\label{sec:examplesArithmetic}
This section provides examples of functions that will do
arithmetic on quantum values. The primary source of these 
algorithms is \cite{Vedral:1995ga}, although the 
algorithms and types used in modular arithmetic are not ones
 I have encountered yet
in the literature.


\subsection{Quantum adder}\label{appsubsec:quantumadder}
This section provides subroutines that perform \emph{carry-save}
arithmetic on \qubits{}. The \inlqpl{carry} and \inlqpl{sum}
routines in \vref{fig:carrysum} function as gates on four \qubits{}
and three \qubits{} respectively. 


\begin{figure}[htbp]
\lstinputlisting[style=linqpl]{examplecode/arithmetic/carrysumgates.qpl}
\caption{\lqpl{} code to implement carry and sum gates}\label{fig:carrysum}
\end{figure}

Additionally, the reverse of the \inlqpl{carry} is 
also defined in the same figure. In order to define a subtraction, 
which can be defined as the reverse of the add function, 
we would also need to define the reverse of the \inlqpl{sum} function.

The addition algorithm adds two lists of \qubits{} and an input
carried \qubit. The first list
is unchanged by the algorithm and the second list is changed to hold the 
sum of the lists, as shown in \vref{fig:quantumadder}.

\begin{figure}[htbp]
\lstinputlisting[style=linqpl]{examplecode/arithmetic/adder.qpl}
\caption{\lqpl{} code to add two lists of \qubits}\label{fig:quantumadder}
\end{figure}

The program proceeds down the lists $A$ and $B$ of input \qubits, first applying
the \inlqpl{carry} to the input carried \qubit, the heads of $A$ and $B$ and
a new zeroed \qubit, $c_1$. When the ends of the lists are reached, a controlled
not and the \inlqpl{sum} are applied. The output $A+B$ list is then started
with $c_1$. Otherwise, the program recurses, calling itself
with $c_1$ and the tails of the input lists. When that returns, \inlqpl{carry}
and \inlqpl{sum} are applied, the results are ``Consed'' to the existing
tails of the lists, $c_1$ is discarded and the program returns.

\subsection{Modular arithmetic}\label{subsec:examplemodarith}
Most treatments of quantum modular arithmetic assume the program / algorithm
will work with a quantum register with a sufficiently large number of \qubits,
as in \cite{Vedral:1995ga}.
In \lqpl, we define a new datatype for quantum modular integers, called
\inlqpl{QuintMod}.  A \inlqpl{QuintMod} consists of a triple of two integers 
and a list of \qubits. The first integer is the maximum size of the list and 
the second integer is the modulus.


\begin{figure}[htbp]
\lstinputlisting[style=linqpl]{examplecode/quintmod/QuintMods.qpl}
\caption{\lqpl{} definitions of QuintMod}\label{fig:quintmods}
\end{figure}

The code for the type definition and conversion from and to integers is 
given in \ref{fig:quintmods}. 

Given these definitions and code for adding and subtracting lists
of \qubits (as in \ref{appsubsec:quantumadder}), it is now
possible to define a \emph{smart constructor} for a \inlqpl{QuintMod}. 

A \inlqpl{QuintMod} triplet has only one invariant that must be maintained. 
That is the list of \qubits{} must have length less than or equal to 
the first number of the triplet. Note this implies that the number represented
by a \qubit{} list may be outside the range of the modulus. For example, 
assume a modulus of $5$. This implies a length of $3$ \qubits. The (qu)bit 
sequences of $101,\ 011,\ 111$, representing $5,6$ and $7$ are allowed.
When converting back to a viewable integer, these would be converted to 
$0,1,2$ respectively.


\begin{figure}[htbp]
\lstinputlisting[style=linqpl]{examplecode/quintmod/smartConstructor.qpl}
\caption{Semi-Smart constructor for QuintMod}\label{fig:makequintmods}
\end{figure}

The function \inlqpl{makeQuint} defined in \ref{fig:makequintmods} maintains
the length invariant, \emph{assuming} that the length is no more than $1$ greater
than allowed. That is, if working with \inlqpl{QuintMod} numbers of length
$n$, the \inlqpl{List(Qubit)} argument has length at most $n + 1$. 

This will imply the represented number passed in is at most $2^{n+1} -1$. As
the modulus is at least $2^{n-1}$, at most two subtractions of the 
modulus will ensure the passed in list represents a number in the range
$0$ -- $2^n - 1$. The subtractions are controlled by the $n+1$st \qubit{}
in the list. 

The reason the numbers may take on the full range is due to the representation.
For example, if we are working in $\mod{4}$, which requires $3$ \qubits{}
 for representation we may attempt to add $7$ to itself. This will 
result in $14$, requiring $4$ \qubits. The final \qubit, recalling we store
numbers with least significant \qubit{} first, will be \ket{1}. Subtracting
$4$ once leaves us with $10$, still requiring $4$ \qubits. Subtracting $4$
once more gives us $6$ which brings us back to the correct range.


\begin{figure}[htbp]
\lstinputlisting[style=linqpl]{examplecode/quintmod/basicArith.qpl}
\caption{Quantum modular addition}\label{fig:quantummodularadd}
\end{figure}

\Ref{fig:quantummodularadd} shows the \lqpl{} code for implementing
modular addition. The function \inlqpl{addM} takes in two 
\inlqpl{QuintMod} elements and adds the first to the second. Note the
use of the smart constructor \inlqpl{makeQuint} to produce the result.


\begin{figure}[htbp]
\lstinputlisting[style=linqpl]{examplecode/quintmod/multSupport.qpl}
\caption{Support functions for Quantum modular multiplication}\label{fig:quantummodularmultsupport}
\end{figure}

At this stage modular multiplication is straightforward to write. 
First, the support functions \inlqpl{ctlCopy} and \inlqpl{ctlDouble} are
defined  in \ref{fig:quantummodularmultsupport}. 
The \inlqpl{ctlCopy} function
creates a ``copy'' of a \inlqpl{QuintMod}. Note that the \qubits
are all created by controlled-Nots of the original list, so the
``copy'' is entangled with the original. If the control \qubit{} is
\ket{0}, the resulting \inlqpl{QuintMod} is zero. The \inlqpl{ctlDouble}
function uses the controlled copying and introduces a new \ket{0} at the
beginning of the list of \qubits{} in the \inlqpl{QuintMod}. This
effectively doubles the number by shifting it one position to the right.
The resulting \inlqpl{QuintMod} is created using the smart constructor 
defined previously.



\begin{figure}[htbp]
\lstinputlisting[style=linqpl]{examplecode/quintmod/multiply.qpl}
\caption{Quantum modular multiplication}\label{fig:quantummodularmultiply}
\end{figure}

The function  \inlqpl{multiplyM} is then
defined as a recursive function in \ref{fig:quantummodularmultiply}.  The
algorithm used  is the standard grade school multiplication method.



\begin{figure}[htbp]
\lstinputlisting[style=linqpl]{examplecode/quintmod/powerSupport.qpl}
\caption{Quantum modular exponentiation support}\label{fig:quantummodularexpsupport}
\end{figure}


Modular exponentiation is written in a similar manner. 
First the support functions \inlqpl{ctlCopyOne} and \inlqpl{square} are
defined in \ref{fig:quantummodularexpsupport}. \inlqpl{ctlCopyOne} 
uses \inlqpl{ctlCopy} defined above to create a copy of a \inlqpl{QuintMod},
but adds $1$ to the resulting \inlqpl{QuintMod} when the 
control \qubit{} is \ket{0}, resulting in the identity for multiplication
rather than the identity for addition. The function \inlqpl{square} 
corresponds to \inlqpl{ctlDouble} in the multiplication case. 


\begin{figure}[htbp]
\lstinputlisting[style=linqpl]{examplecode/quintmod/power.qpl}
\caption{Quantum modular exponentiation}\label{fig:quantummodularexp}
\end{figure}

 The code
for the actual \inlqpl{powerM} function is in \ref{fig:quantummodularexp}.

