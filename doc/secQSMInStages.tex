\section{Quantum stack machine in stages}\label{sec:qsmstate}
The quantum stack machine is described in terms of four 
progressively more elaborate stages. The first stage is
the   \emph{basic QS-machine}, labelled \bms. This stage provides
 facilities for the majority of operations
of our machine, including classical operations, adding and discarding data
 and classical control. The second stage, the \emph{labelled QS-machine},
called \lbms{} adds the capability of applying 
unitary transforms with the modifiers 
\semins{Left, Right} and \semins{IdOnly} as introduced in 
\cite{giles:msc2}. 
The third stage, the  \emph{controlled QS-machine}, is labelled \cms{} and 
provides the
ability to do quantum control. The final stage,  the
\emph{QS-machine}, is labelled \ms{} and 
 adds the ability to call subroutines and do 
recursion.

These stages are ordered in terms of complexity and the
operations definable on them. The ordering is:
\[ \bms < \lbms < \cms < \ms\]
When a function is defined on one of the lower stages, it is possible
to lift it to a function on any of the higher stages. 

%The details of the Haskell implementation of 
%these stages and the lifting functions are given in 
%\vref{subsec:QSM:machinedescription}.

\subsection{Basic quantum stack machine}\label{subsec:basicmachinestate}

The quantum stack machine transitions for the quantum instructions 
 are defined  at this stage.
The state of the basic quantum stack machine  has a code stream, $\cd$, a
classical stack, $S$, a  quantum stack, $Q$, a dump, $D$ and a
name supply, $N$.
\begin{equation}
(\cd,S,Q,D,N)\label{eq:minimalmachinestate} \\
\end{equation}

The code, $\cd$,  is a list of machine instructions. Transitions 
effected by these instructions are detailed in 
 \vref{subsec:transitiondiagrams}. English descriptions of the
instructions and what they do are given in 
 \vref{subsec:repauxinstructions}.


The classical stack, $S$, is a standard stack whose items may
be pushed or pulled onto the top of the stack and specific locations 
may be accessed for both reading and updating.
Classical arithmetic and Boolean operations are done with the top
elements of the classical stack. Thus, an add will pop the
top two elements of the classical stack and then push the result 
 on to the top of the stack.


The dump, $D$, is a holding area for intermediate results and returns. This
is used when measuring quantum bits,  using probabilistic data, 
splitting constructed data types and for calling subroutines. Further
details are given in \vref{sec:representationofdump}.


The name supply, $N$, is an integer that is incremented each
time it is used. The name supply is used when binding nodes to 
constructed data nodes. As they are bound, they are renamed to a unique name
generated from the name supply. For further details on this, see 
the transitions for \qsmins{QBind} at \vref{subsec:quantumstacknodecreation}.

\subsection{Labelled quantum stack machine}\label{subsec:labelledmachinestate}

The labelled QS-machine, designated as \lbms, extends  \bms{} by
labelling the quantum stack, $L(Q)$. The quantum stack is labelled to 
control the application of
 unitary transformations, which allows quantum control to
be implemented.

The labelled QS-machines state is a tuple of five elements:
\begin{equation}
(\cd,S,L(Q),D,N)\label{eq:minimalmachinestateplus}
\end{equation}

The quantum stack is labelled by one of four labels: 
\texttt{Full, Right, Left} or 
\texttt{IdOnly}. These labels 
describe how unitary transformations will be applied to  the quantum stack.

When this labelling was introduced in \cite{giles:msc2}, it 
was used as an instruction modifier rather than a labelling of the
quantum stack. While the implementation of these modifiers  is
changed, the effect on the quantum stack is the same. The quantum stack
machine transitions for unitary transformations are detailed 
in  \vref{subsec:trans:unitarytransformations}.


\subsection{Controlled quantum stack machine}\label{subsec:controlledmachinestate}

The controlled quantum stack machine, \cms, adds the capability to 
 add or remove quantum control.
This stage  adds a control stack, $C$, and changes the 
 tuple of classical stack, labelled quantum stack, dump
and name supply into a  list of tuples of these elements. In the 
machine states, a list will be denoted by enclosing the list items
or types in square brackets.

The \cms{} state is a tuple of three elements, where the third element
is a list of four-tuples:
\begin{equation}
(\cd, C , [(S,L(Q),D,N)]) \label{eq:controlmachinestate}
\end{equation}



The control stack is implemented using 
a list of functions, each of which is defined on the third element of
\cms. The functions in the control stack
transform the list of  tuples $(S,L(Q),D,N)$.  Control points are
added to the control stack 
by placing an identity function at the top of
the stack. Control points are removed by taking the top of the control stack
and applying it to the current third element of \cms, resulting in
a new list of tuples.
Adding a \qubit{} to control will modify the function on top of the control
stack
and change the list of tuples of $(S,L(Q),D,N)$.

\subsection{The complete quantum stack machine}\label{subsec:machinestate}

The complete machine, \ms,  allows the  implementation of  subroutine calling.
Its state consists of 
an infinite list of \cms{} elements.

\begin{equation}
\Inflist{(\cd, C , [(S,L(Q),D,N)])} \label{eq:machinestate}
\end{equation}


Subroutine calling is done in an iterative manner. At the head of the
infinite list, no subroutines are called, but result in divergence.
Divergence is represented in the quantum stack machine by a quantum 
stack with a trace of $0$.

In the next position of the infinite list, a subroutine will be entered once.
If the subroutine
 is recursive or calls other subroutines, those calls will diverge. 
The next position of the infinite list will call one more
level. At the $n^{\mathrm{th}}$ position of the infinite list 
subroutines are executed to a call depth of $n$.

\subsection{The classical stack}\label{subsec:repauxclassicalstack}
The machine uses and creates values
on the classical stack when performing arithmetic operations. 
This object is a standard 
push-down stack with random access.
Currently it accommodates both integer and Boolean values.

\subsection{Representation of the dump}\label{sec:representationofdump}
When processing various operations in the machine, such as  those
labelled as quantum control (measure et. al.), the machine will 
need to save intermediate stack states and results. To illustrate, when 
processing a case deconstruction of a datatype,
the machine saves all partial trees of the node on the dump together
with an empty stack to accumulate
 the results of processing these partial trees.
After processing  each case the current quantum stack is
merged with the result stack and the next partial stack is processed. 
The classical stack is also saved in the
dump element at
the beginning of the process and reset to this saved value 
when each case is evaluated.


The dump is  a list of \emph{dump elements}. There are two distinct types
of dump elements, one for quantum control instructions and one used for
  call statements. The details of these elements may
be found in the description of the quantum control transitions in
 \vref{subsec:measurementandchoice} and 
function calling in \vref{subsec:functioncalling}.

\subsection{Name supply}
The name supply is a read-only register of the machine. It 
provides a unique name
when binding nodes to a data node. The implementation 
uses an integer value which
is incremented for each of the variables in a selection pattern in
a case statement. It is reset to zero at
the start of each program.
