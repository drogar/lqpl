%\floatstyle{boxed}
%\newfloat{QuantumCircuit}{htp}{qci}[chapter]
%\floatname{QuantumCircuit}{Circuit}

%\setlength{\textheight}{8.5in}
%\setlength{\textwidth}{6.5in}
%\setlength{\oddsidemargin}{-.3in}
%\setlength{\evensidemargin}{-.3in}
% macros for processing of the literate files

% reference formats -
\labelformat{chapter}{chapter~#1}
\labelformat{section}{section~#1}
\labelformat{subsection}{sub-section~#1}
\labelformat{subsubsection}{sub-sub-section~#1}
\labelformat{equation}{equation~(#1)}
\labelformat{table}{table~#1}
\labelformat{figure}{figure~#1}


\DefineVerbatimEnvironment%
 {happycodefirst}{Verbatim}{gobble=2,numbers=left,firstnumber=1,fontsize=\footnotesize}
\DefineVerbatimEnvironment%
 {qplcodethesis}{Verbatim}{commandchars=\\\{\},numbers=left,firstnumber=1,fontsize=\footnotesize}
\DefineVerbatimEnvironment%
 {bnf}{Verbatim}{commandchars=\\\{\},fontfamily=courier,fontsize=\footnotesize}
\DefineVerbatimEnvironment%
 {happycodecont}{Verbatim}{gobble=2,numbers=left,firstnumber=last,fontsize=\footnotesize}
% import and redefinition of general macros
\newcommand{\mca}{\cite{bib:mca:vzg1999}}

\newcommand{\dt}{\emph{Digital Typography}\cite{bib:knuth99}}
\newcommand{\mecomp}{\emph{Computers and TypeSetting: Millennium Edition}\cite{bib:knuth2000}}
\newcommand{\mmix}{\emph{MMIXWare}\cite{bib:knuth2000mx}}
\newcommand{\tplang}{\emph{Theories of Programming Languages}\cite{reynolds1998:theoryprogramminglang}}
\newcommand{\cattype}{\emph{Categories for Types}\cite{bib:categoriesfortypes:crole1993}}
\newcommand{\denpl}{\emph{The Denotational Description of Programming Languages}\cite{gordon79:_denot_descr_progr_languag}}
\newcommand{\lfl}{\emph{\LaTeX for Linux}\cite{bib:lipkin99}}
\newcommand{\taoone}{\emph{Fundamental Algorithms}\cite{bib:fundalogs:knuth97}}
\newcommand{\taotwo}{\emph{Seminumerical Algorithms}\cite{bib:seminumerical:knuth98}}
\newcommand{\citett}{\cite{bib:seminumerical:knuth98}}
\newcommand{\greektb}{\emph{Ancient Greek Alive}\cite{bib:saffire99}}
\newcommand{\fermat}{\emph{Fundamental Algorithms}\cite{bib:singh97}}
\newcommand{\dovedd}{\emph{Darkness Descending}\cite{bib:turtledove00}}
\newcommand{\grpassman}{\emph{The Algebraic Structure of Group Rings}\cite{bib:passman77}}
\newcommand{\latex}{\LaTeX}
\newcommand{\latexgrcomp}{\emph{The \LaTeX Graphics Companion}\cite{bib:goosens97}}
\newcommand{\latexref}{\emph{\LaTeX: A Document Preperation System}\cite{bib:lambert85}}
\newcommand{\progpython}{\emph{Programming Python}\cite{bib:lutz96}}
\newcommand{\handplone}{\emph{Handbook of Programming Languages, Volume I:
Object Oriented Languages}\cite{bib:salus98v1}}
\newcommand{\handplfour}{\emph{Handbook of Programming Languages, Volume IV:
Functional and Logic Programming Languages}\cite{bib:salus98v4}}
\newcommand{\domainandlambda}{\emph{Domains and Lambda-Calculi}\cite{amadio98:domainsandlambda}}
\newcommand{\fpintro}{\emph{Introduction to Functional Programming using Haskell}%
\cite{Bird1998:introtoFPusingHaskell}}
\newcommand{\fpdata}{\emph{Purely Functional Data Structures}\cite{bib:Okasaki:PFData}}
\newcommand{\fprogio}{\emph{Functional Programming and Input/Output}%
\cite{bib:gordon94}}
\newcommand{\introhaskel}{\emph{A Gentle Introduction to Haskell 98}%
\cite{bib:Hudak:introhaskel}}
\newcommand{\compcomplexity}{\emph{Computational Complexity}%
\cite{bib:compcomplexity:papa1994}}
\newcommand{\gentlect}{\emph{A Gentle introduction to Category Theory}%
\cite{bib:gentleCategoryTheory:mmf92b}}



\newcommand{\nameref}[1]{\textbf{#1}}
\newcommand{\unixcmd}[1]{\texttt{#1}}

\newcommand{\incsec}[1]{\subsection{#1}}

\newcommand{\bi}{\begin{itemize}}
\newcommand{\ei}{\end{itemize}}
\newcommand{\be}{\begin{enumerate}}
\newcommand{\ee}{\end{enumerate}}
\newcommand{\bd}{\begin{description}}
\newcommand{\ed}{\end{description}}
\newcommand{\itembf}[1]{\item{\textbf{#1}}}
\newcommand{\itemem}[1]{\item{\emph{#1}}}
\newcommand{\itemtt}[1]{\item{\texttt{#1}}}

\newcommand{\itemhyp}[1]{\item{\textbf{#1:}}\hypertarget{#1}{}}
\newcommand{\itemref}[1]{\hyperlink{#1}{#1}}

\renewcommand{\fullref}[1]{Section \ref{#1} on page \pageref{#1}}

\newcommand{\LG}{\ensuremath{L^1(G)}}

\newcommand{\supp}{\mathrm{supp}}
\newcommand{\KG}{\ensuremath{KG}}
\newcommand{\integers}{\ensuremath{ \mathbb{Z}}}
\newcommand{\naturals}{\ensuremath{ \mathbb{N}}}

\newcommand{\burnat}[1]{\ensuremath{ \mathbb{N}(#1)}}
\newcommand{\NA}{\burnat{A}}
\newcommand{\NNA}{\burnat{\NA}}
\newcommand{\NNNA}{\burnat{\NNA}}
\newcommand{\NB}{\burnat{B}}
\newcommand{\NNB}{\burnat{\NB}}
\newcommand{\NNNB}{\burnat{\NNB}}
\newcommand{\NC}{\burnat{C}}
\newcommand{\NNC}{\burnat{\NC}}
\newcommand{\NNNC}{\burnat{\NNC}}
\newcommand{\Nf}{\burnat{f}}
\newcommand{\Ng}{\burnat{g}}
\newcommand{\Nh}{\burnat{h}}
\newcommand{\NNf}{\burnat{\Nf}}
\newcommand{\NNg}{\burnat{\Ng}}
\newcommand{\NNh}{\burnat{\Nh}}
\newcommand{\N}{\naturals}
\newcommand{\add}[1]{\ensuremath{\mathsf{add}_{#1}}}
\newcommand{\adda}{\ensuremath{\mathsf{add}_A}}
\newcommand{\addna}{\ensuremath{\mathsf{add}_{\NA}}}
\newcommand{\addb}{\ensuremath{\mathsf{add}_B}}
\newcommand{\addnb}{\ensuremath{\mathsf{add}_{\NB}}}

\newcommand{\rec}[1]{\ensuremath{\mathbb{RC}(#1)}}
\newcommand{\recab}{\rec{A,B}}
\newcommand{\crs}[2]{\ensuremath{#1\times#2}}


\newcommand{\rationals}{\ensuremath{ \mathbb{Q}}}
\newcommand{\complexes}{\ensuremath{ \mathbb{C}}}
\newcommand{\reals}{\ensuremath{ \Re}}
\newcommand{\series}[3]{\ensuremath{#1_{#2}, \ldots, #1_{#3}}}
\newcommand{\pseries}[4]{\ensuremath{#1_{#3}:{#2}_{#3}, \ldots, #1_{#4}:{#2}_{#4}}}
\newcommand{\seriesd}[6]{\ensuremath{#1_{#2}, \ldots, #1_{#3}#4_{#5}, \ldots, #4_{#6}}}
\newcommand{\BigO}[1]{\ensuremath{\mathscr{O}({#1})}}
\newcommand{\etape}{\ensuremath{\sqcup\sqcup\ldots}}
\newcommand{\di}[1]{{\bfseries \itshape {#1}}}
\newcommand{\czero}{\ensuremath{\mathscr{C}_0}}
\newcommand{\cone}{\ensuremath{\mathscr{C}_1}}
\newcommand{\dzero}[1]{\ensuremath{D_0(#1)}}
\newcommand{\done}[1]{\ensuremath{D_1(#1)}}
\newcommand{\fab}{\ensuremath{f:A\to{}B}}
\newcommand{\gbc}{\ensuremath{g:B\to{}C}}
\newcommand{\hcd}{\ensuremath{h:C\to{}D}}

% for 613
\newcommand{\ora}[1]{\ensuremath{\overrightarrow{#1}}}
\newcommand{\ola}[1]{\ensuremath{\overleftarrow{#1}}}
\newcommand{\entails}[1]{\vdash^{#1}}
\newcommand{\nat}[1]{\{\text{succ}\colonn{#1},\text{zero}\colonn{()}\}}
\newcommand{\munat}{\mu{}x.\nat{x}}
\newcommand{\colonn}[1]{\coln#1}
%\newcommand{\colon}{\coln\!\!}
\newcommand{\coln}{\!\!:\!\!}
\newcommand{\dcolon}{\!\!::\!\!}
\newcommand{\banl}{\{\!|}
\newcommand{\banr}{|\!\}}
\newcommand{\fold}[2]{\ensuremath{\banl #1 | #2 \banr}}
%\newcommand{\implies}{\ensuremath{\Rightarrow}}
\renewcommand{\L}[1]{\ensuremath{\mathbb{L}(#1)}}
\newcommand{\LA}{\L{A}}

\newcommand{\Leaf}{\ensuremath{\text{\footnotesize\textsc{Leaf}}}}
\newcommand{\Node}{\ensuremath{\text{\footnotesize\textsc{Node}}}}
\newcommand{\clt}{\prgtransfunc{clt}}
\newcommand{\Zero}{\ensuremath{\text{\footnotesize\textsc{Zero}}}}
\newcommand{\Succ}{\ensuremath{\text{\footnotesize\textsc{Succ}}}}
\newcommand{\muzero}{\ensuremath{\ola{\Zero}}}
\newcommand{\musucc}{\ensuremath{\ola{\Succ}\cdot}}
\newcommand{\prgtransfunc}[1]{\ensuremath{\text{#1}}}
\newcommand{\zip}{\prgtransfunc{zip}}
\newcommand{\unzip}{\prgtransfunc{unzip}}
\newcommand{\zp}{\prgtransfunc{zp}}
\newcommand{\unz}{\prgtransfunc{unz}}
\newcommand{\nil}{\prgtransfunc{nil}}
\newcommand{\cons}{\prgtransfunc{cons}}
\newcommand{\tl}{\prgtransfunc{tl}}
\newcommand{\hd}{\prgtransfunc{hd}}
\newcommand{\rev}{\prgtransfunc{rev}}
\newcommand{\frev}{\prgtransfunc{frev}}
\newcommand{\acc}{\prgtransfunc{acc}}
\newcommand{\evens}{\prgtransfunc{evens}}
\newcommand{\evn}{\prgtransfunc{evn}}
\newcommand{\dbl}{\prgtransfunc{dbl}}
\newcommand{\get}{\prgtransfunc{get}}
\newcommand{\gtt}{\prgtransfunc{gtt}}
\newcommand{\gttc}[1]{\gtt\llbracket #1 \rrbracket}
\newcommand{\app}{\prgtransfunc{app}}
\newcommand{\appc}[1]{\app\llbracket #1 \rrbracket}
\newcommand{\munil}{\ensuremath{\ola{\nil}}}
\newcommand{\mucons}{\ensuremath{\ola{\cons}}}
\newcommand{\tlnu}{\ensuremath{\ora{\tl}\cdot}}
\newcommand{\hdnu}{\ensuremath{\ora{\hd}\cdot}}


\newcommand{\il}[1]{\ensuremath{\mathbb{IL}(#1)}}
\newcommand{\ila}{\il{A}}
\newcommand{\ilila}{\il{\ila}}
\newcommand{\ils}{\ensuremath{[a_1,a_2,a_3,\ldots]}}
\newcommand{\ilss}[1]{\ensuremath{a_{#1 1},a_{#1 2},a_{#1 3},\ldots]}}
\newcommand{\ilmataa}{\ensuremath{\begin{bmatrix} [a_{11},a_{12},a_{13},\ldots]\\
 [a_{21},a_{22},a_{23},\ldots]\\
 [a_{31},a_{32},a_{33},\ldots]\\
\vdots
\end{bmatrix}}}
\newcommand{\ilmataatr}{\ensuremath{\begin{bmatrix} [a_{11},a_{21},a_{31},\ldots]\\
 [a_{12},a_{22},a_{32},\ldots]\\
 [a_{13},a_{23},a_{33},\ldots]\\
\vdots
\end{bmatrix}}}



\newcommand{\nm}{\ensuremath{n\times{}m}}
\newcommand{\mn}{\ensuremath{m\times{}n}}

\newcommand{\specialcat}[1]{\textsc{#1}}
\newcommand{\ltrcat}[1]{\ensuremath{\mathfrak{#1}}}
\newcommand{\B}{\ltrcat{B}}
\newcommand{\C}{\ltrcat{C}}
\newcommand{\X}{\ltrcat{X}}
\newcommand{\sets}{\specialcat{Sets}}
\newcommand{\rel}{\specialcat{Rel}}
\newcommand{\mon}{\specialcat{Mon}}
\newcommand{\ring}{\specialcat{Rng}}
\newcommand{\cring}{\specialcat{CRng}}
\newcommand{\cat}{\textbf{\specialcat{Cat}}}
\newcommand{\poset}{\specialcat{Poset}}
\newcommand{\preorder}{\specialcat{Preorder}}


\newcommand{\obj}[1]{\ensuremath{#1_{obj}}}
\newcommand{\bottom}[1]{\perp_{#1}}
\newcommand{\finpower}{\mathscr{P}_{fin}}

\newcommand{\category}[4]{%
\begin{description}%
\item{\textbf{Objects: }}{#1}%
\item{\textbf{Maps: }}{#2}%
\item{\textbf{Identity: }}{#3}%
\item{\textbf{Composition: }}{#4}%
 \end{description}%
%
}

\newcommand{\pfcategory}[3]{%
\begin{description}%
\item{\textbf{Well-Defined: }}{#1}%
\item{\textbf{Identities: }}{#2}%
\item{\textbf{Associativity: }}{#3}%
 \end{description}%
%
}
%\newcommand{\categoryp}[4]{%
%\paragraph{Objects:}{#1}\\%
 %
%\paragraph{Maps:}{#2}\\%
 %
%\paragraph{Identity:}{#3}\\%
 %
%\paragraph{Composition:}{#4}\\%
 %
% }
\newenvironment{tageqnarray*}{\begin{equation*}\begin{array}{rclr}}{\end{array}\end{equation*}}

\newcounter{cdiagram}[section]
\renewcommand{\thecdiagram}{%
\thesection.\arabic{cdiagram}
}

\newcommand{\dicapt}[2]{%
{%
\refstepcounter{cdiagram}%
\begin{center}%
\thecdiagram\quad{}#1\label{#2} %
\end{center}%
}
}
\newcounter{annctr}
\newsavebox{\AnnpBox}
\newenvironment{annproof}
{\setcounter{annctr}{16}}
{\usebox{\AnnpBox}}

\newcommand{\annp}[1]{%
\stepcounter{annctr}%
\sbox{\AnnpBox}{\linebreak \theannctr{}.\quad{}#1}%
\theannctr%
}

\newcommand{\entailf}[1]{\vdash^{\annp{#1}}}


%\newcounter{cdiagram}[section]
%\newcommand{\dicapt}[2]{%
%\begin{center}%
%#1 #2 %
%\end{center}%


\newtheorem{theorem}{Theorem}[section]
\newtheorem{definition}[theorem]{Definition}
\newtheorem{notation}[theorem]{Notation}
\newtheorem{example}{Example}
\newtheorem{exercise}{Exercise}
\newtheorem{lemma}[theorem]{Lemma}
%\newtheorem*{cor}{Corollary}
%\newtheorem*{vhyp}{Valiant's Hypothesis}

\newcommand{\cclass}[2]{\ensuremath{\mathrm{#1}(#2)}}
\newcommand{\Time}[1]{\cclass{TIME}{#1}}
\newcommand{\Ntime}[1]{\cclass{NTIME}{#1}}
\newcommand{\Space}[1]{\cclass{SPACE}{#1}}
\newcommand{\Nspace}[1]{\cclass{NSPACE}{#1}}

% general math symbols
\newcommand{\union}{\ensuremath{\bigcup}}
\newcommand{\disjunion}{\ensuremath{\sqcup}}
\newcommand{\intersect}{\ensuremath{\bigcap}}
\newcommand{\logor}{\ensuremath{\lor}}
\newcommand{\logand}{\ensuremath{\land}}
\newcommand{\lognand}{\ensuremath{\barwedge}}
\newcommand{\natmap}{\ensuremath{\Rightarrow}}
\newcommand{\fctrmap}{\ensuremath{\to}}
\newcommand{\fnctrmap}{\fctrmap}
\newcommand{\fmap}{\ensuremath{\to}}
\newcommand{\produces}{\ensuremath{\to}}
\newcommand{\ladjoint}{\ensuremath{\dashv}}
\newcommand{\pproj}{\ensuremath{\preceq_p}}


\numberwithin{equation}{section}
\newenvironment{coproduct}{\left\{ \begin{array}{l}} {\end{array} \right\} }
%\newcommand{\coproductthree}[3]{\ensuremath{
%\left\{ \begin{array}{l} #1 \\ #2 \\#3 \end{array} \right\} }}
%\newcommand{\coproduct}[2]{\ensuremath{
%\left\{ \begin{array}{l} #1 \\ #2  \end{array} \right\} }}
\newcommand{\product}[2]{\ensuremath{
\left( \begin{array}{l} #1 \\ #2  \end{array} \right) }}
\newcommand{\inductf}[3]{\ensuremath{
\left[ \begin{array}{l}  \# #1\mapsto #2 (#3)  \end{array} \right]}}
\newcommand{\coinductf}[2]{\ensuremath{
\left[ \begin{array}{l}  \# #1\mapsto #2   \end{array} \right]}}

% adding commands for prooftheory
\newcommand{\context}[3] {%
\ensuremath{%
\begin{array}[b]{c}%
  \begin{array}[b]{|c|}%
  \hline %
   \begin{array}[b]{c}
     #2\quad%
   \end{array}\\
  \hline \\%
  #3\\%
  \hline%
  \end{array} \\%
#1 \\%
\end{array}%
}%
}

\newcommand{\muInfer}[4] {%
\ensuremath{%
\begin{array}[b]{c}%
  \begin{array}[b]{|c}%
  \hline %
   \begin{array}[b]{c|c}
     #2\quad & \quad#3%
   \end{array}\\
  \hline \\%
  #4\\%
  \hline%
  \end{array} \\%
#1 \\%
\end{array}%
}%
}


\newcommand{\muInferC}[5] {%
\ensuremath{%
\begin{array}[b]{c}%
  \begin{array}[b]{|c}%
  \hline %
   \begin{array}[b]{c|c|c}
     #2\quad & \quad#3&\quad#4%
   \end{array}\\
  \hline \\%
  #5\\%
  \hline%
  \end{array} \\%
#1 \\%
\end{array}%
}%
}

\newcommand{\nuInfer}[4] {
\ensuremath{
\begin{array}[b]{c}
  \begin{array}[b]{c|}
  \hline
   \begin{array}[b]{c|c}
     #2\quad & \quad#3%
   \end{array}\\
  \hline \\
  #4 \\
  \hline
  \end{array} \\
#1 \\
\end{array}
}
}


\newcommand{\nuInferC}[5] {
\ensuremath{
\begin{array}[b]{c}
  \begin{array}[b]{c|}
  \hline
   \begin{array}[b]{c|c|c}
     #2\quad & \quad#3&\quad#4%
   \end{array}\\
  \hline \\
  #5 \\
  \hline
  \end{array} \\
#1 \\
\end{array}
}
}



\newcommand{\cde}[1]{\texttt{#1}}
\newcommand{\kword}[1]{\texttt{\textbf{#1}}}
\newcommand{\hsktype}[1]{\texttt{\textbf{#1}}}
\newcommand{\hskfnc}[1]{\textit{\textbf{#1}}}
\newenvironment{code}{\small\verbatim}{\normalsize\endverbatim}


%adding commands for denotational semantics.
\newcommand{\den}[1]{\ensuremath{[\![#1]\!]}}
\newcommand{\densub}[2]{\ensuremath{[\![#2]\!]_{#1}}}




\newcommand{\inflist}[1]{\ensuremath{\mathbb{IL}({#1})}}
\renewcommand{\incsec}[1]{\subsection{#1}}
% end of redefinitions.
\newcommand{\incsubsec}[1]{\subsubsection{#1}}
\newcommand{\incsubsubsec}[1]{\paragraph{#1}}
\newcommand{\qplcode}[1]{\textbf{#1}}
\DefineVerbatimEnvironment%
{code}{Verbatim}{numbers=left,fontsize=\footnotesize,firstnumber=1}
\newcommand{\CodeResetNumbers}{\RecustomVerbatimEnvironment%
{code}{Verbatim}{numbers=left,fontsize=\footnotesize,firstnumber=1}}
\newcommand{\CodeContinueNumbers}{\RecustomVerbatimEnvironment%
{code}{Verbatim}{numbers=left,fontsize=\footnotesize,firstnumber=last}}
\renewcommand{\C}{\ensuremath{\mathbb{C}}}

%    Q-circuit version 1.06
%    Copyright (C) 2004  Steve Flammia & Bryan Eastin

%    This program is free software; you can redistribute it and/or modify
%    it under the terms of the GNU General Public License as published by
%    the Free Software Foundation; either version 2 of the License, or
%    (at your option) any later version.
%
%    This program is distributed in the hope that it will be useful,
%    but WITHOUT ANY WARRANTY; without even the implied warranty of
%    MERCHANTABILITY or FITNESS FOR A PARTICULAR PURPOSE.  See the
%    GNU General Public License for more details.
%
%    You should have received a copy of the GNU General Public License
%    along with this program; if not, write to the Free Software
%    Foundation, Inc., 59 Temple Place, Suite 330, Boston, MA  02111-1307  USA

%\usepackage[matrix,frame,arrow]{xy}
\usepackage{amsmath}
\newcommand{\bra}[1]{\ensuremath{\left\langle{#1}\right\vert}}
\newcommand{\ket}[1]{\ensuremath{\left\vert{#1}\right\rangle}}
    % Defines Dirac notation.
\newcommand{\qw}[1][-1]{\ar @{-} [0,#1]}
    % Defines a wire that connects horizontally.  By default it connects to the object on the left of the current object.
    % WARNING: Wire commands must appear after the gate in any given entry.
\newcommand{\qwx}[1][-1]{\ar @{-} [#1,0]}
    % Defines a wire that connects vertically.  By default it connects to the object above the current object.
    % WARNING: Wire commands must appear after the gate in any given entry.
\newcommand{\cw}[1][-1]{\ar @{=} [0,#1]}
    % Defines a classical wire that connects horizontally.  By default it connects to the object on the left of the current object.
    % WARNING: Wire commands must appear after the gate in any given entry.
\newcommand{\cwx}[1][-1]{\ar @{=} [#1,0]}
    % Defines a classical wire that connects vertically.  By default it connects to the object above the current object.
    % WARNING: Wire commands must appear after the gate in any given entry.
\newcommand{\gate}[1]{*{\xy *+<.6em>{#1};p\save+LU;+RU **\dir{-}\restore\save+RU;+RD **\dir{-}\restore\save+RD;+LD **\dir{-}\restore\POS+LD;+LU **\dir{-}\endxy} \qw}
    % Boxes the argument, making a gate.
\newcommand{\meter}{\gate{\xy *!<0em,1.1em>h\cir<1.1em>{ur_dr},!U-<0em,.4em>;p+<.5em,.9em> **h\dir{-} \POS <-.6em,.4em> *{},<.6em,-.4em> *{} \endxy}}
    % Inserts a measurement meter.
\newcommand{\measure}[1]{*+[F-:<.9em>]{#1} \qw}
    % Inserts a measurement bubble with user defined text.
\newcommand{\measuretab}[1]{*{\xy *+<.6em>{#1};p\save+LU;+RU **\dir{-}\restore\save+RU;+RD **\dir{-}\restore\save+RD;+LD **\dir{-}\restore\save+LD;+LC-<.5em,0em> **\dir{-} \restore\POS+LU;+LC-<.5em,0em> **\dir{-} \endxy} \qw}
    % Inserts a measurement tab with user defined text.
\newcommand{\measureD}[1]{*{\xy*+=+<.5em>{\vphantom{#1}}*\cir{r_l};p\save*!R{#1} \restore\save+UC;+UC-<.5em,0em>*!R{\hphantom{#1}}+L **\dir{-} \restore\save+DC;+DC-<.5em,0em>*!R{\hphantom{#1}}+L **\dir{-} \restore\POS+UC-<.5em,0em>*!R{\hphantom{#1}}+L;+DC-<.5em,0em>*!R{\hphantom{#1}}+L **\dir{-} \endxy} \qw}
    % Inserts a D-shaped measurement gate with user defined text.
\newcommand{\multimeasure}[2]{*+<1em,.9em>{\hphantom{#2}} \qw \POS[0,0].[#1,0];p !C *{#2},p \drop\frm<.9em>{-}}
    % Draws a multiple qubit measurement bubble starting at the current position and spanning #1 additional gates below.
    % #2 gives the label for the gate.
    % You must use an argument of the same width as #2 in \ghost for the wires to connect properly on the lower lines.
\newcommand{\multimeasureD}[2]{*+<1em,.9em>{\hphantom{#2}}\save[0,0].[#1,0];p\save !C *{#2},p+LU+<0em,0em>;+RU+<-.8em,0em> **\dir{-}\restore\save +LD;+LU **\dir{-}\restore\save +LD;+RD-<.8em,0em> **\dir{-} \restore\save +RD+<0em,.8em>;+RU-<0em,.8em> **\dir{-} \restore \POS !UR*!UR{\cir<.9em>{r_d}};!DR*!DR{\cir<.9em>{d_l}}\restore \qw}
    % Draws a multiple qubit D-shaped measurement gate starting at the current position and spanning #1 additional gates below.
    % #2 gives the label for the gate.
    % You must use an argument of the same width as #2 in \ghost for the wires to connect properly on the lower lines.
\newcommand{\control}{*-=-{\bullet}}
    % Inserts an unconnected control.
\newcommand{\controlo}{*!<0em,.04em>-<.07em,.11em>{\xy *=<.45em>[o][F]{}\endxy}}
    % Inserts a unconnected control-on-0.
\newcommand{\ctrl}[1]{\control \qwx[#1] \qw}
    % Inserts a control and connects it to the object #1 wires below.
\newcommand{\ctrlo}[1]{\controlo \qwx[#1] \qw}
    % Inserts a control-on-0 and connects it to the object #1 wires below.
\newcommand{\targ}{*{\xy{<0em,0em>*{} \ar @{ - } +<.4em,0em> \ar @{ - } -<.4em,0em> \ar @{ - } +<0em,.4em> \ar @{ - } -<0em,.4em>},*+<.8em>\frm{o}\endxy} \qw}
    % Inserts a CNOT target.
\newcommand{\qswap}{*=<0em>{\times} \qw}
    % Inserts half a swap gate. 
    % Must be connected to the other swap with \qwx.
\newcommand{\multigate}[2]{*+<1em,.9em>{\hphantom{#2}} \qw \POS[0,0].[#1,0];p !C *{#2},p \save+LU;+RU **\dir{-}\restore\save+RU;+RD **\dir{-}\restore\save+RD;+LD **\dir{-}\restore\save+LD;+LU **\dir{-}\restore}
    % Draws a multiple qubit gate starting at the current position and spanning #1 additional gates below.
    % #2 gives the label for the gate.
    % You must use an argument of the same width as #2 in \ghost for the wires to connect properly on the lower lines.
\newcommand{\ghost}[1]{*+<1em,.9em>{\hphantom{#1}} \qw}
    % Leaves space for \multigate on wires other than the one on which \multigate appears.  Without this command wires will cross your gate.
    % #1 should match the second argument in the corresponding \multigate. 
\newcommand{\push}[1]{*{#1}}
    % Inserts #1, overriding the default that causes entries to have zero size.  This command takes the place of a gate.
    % Like a gate, it must precede any wire commands.
    % \push is useful for forcing columns apart.
    % NOTE: It might be useful to know that a gate is about 1.3 times the height of its contents.  I.e. \gate{M} is 1.3em tall.
    % WARNING: \push must appear before any wire commands and may not appear in an entry with a gate or label.
\newcommand{\gategroup}[6]{\POS"#1,#2"."#3,#2"."#1,#4"."#3,#4"!C*+<#5>\frm{#6}}
    % Constructs a box or bracket enclosing the square block spanning rows #1-#3 and columns=#2-#4.
    % The block is given a margin #5/2, so #5 should be a valid length.
    % #6 can take the following arguments -- or . or _\} or ^\} or \{ or \} or _) or ^) or ( or ) where the first two options yield dashed and
    % dotted boxes respectively, and the last eight options yield bottom, top, left, and right braces of the curly or normal variety.
    % \gategroup can appear at the end of any gate entry, but it's good form to pick one of the corner gates.
    % BUG: \gategroup uses the four corner gates to determine the size of the bounding box.  Other gates may stick out of that box.  See \prop. 
\newcommand{\rstick}[1]{*!L!<-.5em,0em>=<0em>{#1}}
    % Centers the left side of #1 in the cell.  Intended for lining up wire labels.  Note that non-gates have default size zero.
\newcommand{\lstick}[1]{*!R!<.5em,0em>=<0em>{#1}}
    % Centers the right side of #1 in the cell.  Intended for lining up wire labels.  Note that non-gates have default size zero.
\newcommand{\ustick}[1]{*!D!<0em,-.5em>=<0em>{#1}}
    % Centers the bottom of #1 in the cell.  Intended for lining up wire labels.  Note that non-gates have default size zero.
\newcommand{\dstick}[1]{*!U!<0em,.5em>=<0em>{#1}}
    % Centers the top of #1 in the cell.  Intended for lining up wire labels.  Note that non-gates have default size zero.
\newcommand{\Qcircuit}{\xymatrix @*=<0em>}
    % Defines \Qcircuit as an \xymatrix with entries of default size 0em.
 % for quantum circuits

\newcommand{\marginnote}[1]{\marginpar{\tiny #1}}



\lstdefinestyle{hskl}{language=Haskell,
%  basicstyle=\small, % swap this and the following line for prop. font
  basicstyle=\small,
%  keywordstyle=\underbar, % these look ugly
  identifierstyle=\itshape,
  commentstyle=\ttfamily,
%  commentstyle=\underbar
  flexiblecolumns=false,
  basewidth={0.5em,0.45em},
  morekeywords={Map},
  % The following replace compound charcters like ->
  % Something is missing - someone kindly sent me an email which
  % I've lost - but it's obvious how to add more.
  literate={-}{{$-$}}1 {+}{{$+$}}1 {/}{{$/$}}1
     {*}{{$\times$}}1 {=}{{$=$}}1
     {\%}{{$\%$}}1 {=}{{$=$}}1
           {>}{{$>$}}1 {<}{{$<$}}1
           {>>}{{$\gg$}}2 {<<}{{$\ll$}}2
           {=>}{{$\Rightarrow$}}2 {>=}{{$\geq$}}2 {<-}{{$\leftarrow$}}2
           {<=}{{$\leq$}}2
           {<==}{{$\Longleftarrow$}}3
           {=/=}{{$\neq$}}3
}
\lstdefinestyle{origlinqpl}{language=lqpl,basicstyle=\footnotesize\ttfamily,%
  numbers=left,%
        numberstyle=\tiny,%
        numbersep=6pt }
\lstdefinestyle{linqpl}{language=lqpl,basicstyle=\footnotesize,%
  numbers=left,%
        numberstyle=\tiny,%
        numbersep=6pt,%
  escapechar=`,%
  literate={-}{{$-$}}1 {+}{{$+$}}1 {/}{{$/$}}1%
     {*}{{$\times$}}1 {=}{{$=$}}1%
     {\%}{{$\%$}}1 {=}{{$=$}}1%
           {>}{{$>$}}1 {<}{{$<$}}1%
           {>>}{{$\gg$}}2 {<<}{{$\ll$}}2%
           {=>}{{$\Rightarrow$}}2 {>=}{{$\geq$}}2%
           {=<}{{$\leq$}}2%
           {<=}{{$\Leftarrow$}}3%
           {=/=}{{$\neq$}}3 }
\lstdefinestyle{linqplnonum}{language=lqpl,basicstyle=\footnotesize,%
  escapechar=`,%
  literate={-}{{$-$}}1 {+}{{$+$}}1 {/}{{$/$}}1%
     {*}{{$\times$}}1 {=}{{$=$}}1%
     {\%}{{$\%$}}1 {=}{{$=$}}1%
           {>}{{$>$}}1 {<}{{$<$}}1%
           {>>}{{$\gg$}}2 {<<}{{$\ll$}}2%
           {=>}{{$\Rightarrow$}}2 {>=}{{$\geq$}}2%
           {=<}{{$\leq$}}2%
           {<=}{{$\Leftarrow$}}3%
           {=/=}{{$\neq$}}3 }
\lstdefinestyle{inlinqpl}{language=lqpl,basicstyle=\footnotesize\ttfamily,%
  escapechar=`}

\newcommand{\lbl}{{\cdptr}}
\newcommand{\cd}{\ensuremath{\mathcal{C}}}
\newcommand{\lblcd}{{\cd}}
%\newcommand{\lblcd}{\ensuremath{@\lbl}}
\newcommand{\cdptr}{\ensuremath{\triangleright\cd}}
\newcommand{\TODO}[1]{\begin{quote}TODO: \textbf{#1}\end{quote}}
\newcommand{\lqpl}{L-QPL}
\newcommand{\linearqpl}{Linear QPL}

\newcommand{\inlqpl}[1]{\protect{\lstinline[style=linqpl]!#1!}}
\newcommand{\inlhskl}[1]{\lstinline[style=hskl]!#1!}


\newcommand{\Had}{\text{Hadamard}}
\newcommand{\nottr}{\text{Not}}
\newcommand{\Cnot}{controlled{-}\nottr}


%\newcommand{\Z}{\ensuremath{Z}}

\newcommand{\stacknode}[1]{\ensuremath{\mathrm{\texttt{#1}}}}
\newcommand{\Int}{\stacknode{Int}}
\newcommand{\Datatype}{\stacknode{datatype}}
\newcommand{\Bool}{\stacknode{Bool}}

\newcommand{\vc}[1]{\ensuremath{\mathbold{#1}}}
\newcommand{\syntacticset}[1]{\ensuremath{\mathbold{#1}}}

\newcommand{\n}{\syntacticset{N}}
\newcommand{\T}{\syntacticset{T}}
\newcommand{\Q}{\syntacticset{Q}}
\newcommand{\R}{\syntacticset{R}}
\newcommand{\Data}{\syntacticset{Data}}
\newcommand{\Qloc}{\syntacticset{Qloc}}
\newcommand{\Cloc}{\syntacticset{Cloc}}
\newcommand{\Loc}{\syntacticset{Loc}}
\newcommand{\Aexp}{\syntacticset{Aexp}}
\newcommand{\Bexp}{\syntacticset{Bexp}}
\newcommand{\Cexp}{\syntacticset{Cexp}}
\newcommand{\Stm}{\syntacticset{Stm}}
\newcommand{\eval}[3]{\ensuremath{\langle{#1},{#2}\rangle\to{#3}}}
\newcommand{\true}{\ensuremath{\mathbold{true}}}
\newcommand{\false}{\ensuremath{\mathbold{false}}}


\newcommand{\qcons}[1]{\texttt{#1}}
\newcommand{\qtype}[1]{\texttt{\bf{#1}}}
\newcommand{\bms}{\specialcat{Bqsm}}
\newcommand{\lbms}{\specialcat{lBqsm}}
\newcommand{\cms}{\specialcat{Cqsm}}
\newcommand{\ms}{\specialcat{QSM}}

\newcommand{\qubit}{\ensuremath{\mathbold{qubit}}}
\newcommand{\bit}{\ensuremath{\mathbold{bit}}}
\newcommand{\qubits}{\ensuremath{\mathbold{qubits}}}
\newcommand{\bits}{\ensuremath{\mathbold{bits}}}

\newcommand{\complex}{\ensuremath{\mathbb{C}}}

\newcommand{\qbits}{\qubits}

\newcommand{\qsp}{\ensuremath{\to}}

\newcommand{\qsnodeij}[4]{\ensuremath{#1\{{#2}\qsp {#3}\}^{#4}}}

\newcommand{\qsbit}[4]{\ensuremath{#1\{0\qsp {#2};1\qsp {#3}\}^{#4}}}



\newcommand{\qsqubit}[6]{\ensuremath{#1\{00\qsp {#2};01\qsp {#3};10\qsp {#4};11\qsp {#5}\}^{#6}}}

\newcommand{\qsbitUp}[4]{\ensuremath{#1\left\{\begin{array}{l}%
                  0\qsp {#2}\\%
                        1\qsp {#3}%
                        \end{array}%
               \right\}^{#4}}}


\newcommand{\qsqubitUp}[6]{\ensuremath{#1\left\{\begin{array}{ll}%
                  00\qsp {#2};&01\qsp {#3}\\%
                        10\qsp {#4};&11\qsp {#5}%
                        \end{array}%
               \right\}^{#6}}}
\newcommand{\qsqubitAllUp}[6]{\ensuremath{#1\left\{\begin{array}{l}%
                  00\qsp {#2}\\%
                        01\qsp {#3}\\%
                        10\qsp {#4}\\%
                        11\qsp {#5}%
                        \end{array}%
               \right\}^{#6}}}

\newcommand{\eentail}{\ensuremath{\Vdash\!\!\dashv}}
\newcommand{\ientail}{\ensuremath{\Vdash}}
\newcommand{\qentail}{\ensuremath{\vdash}}
\newcommand{\qcontext}{\ensuremath{\, |\, }}
\newcommand{\qcontrolled}[2]{\ensuremath{{#1}\Leftarrow{#2}}}
\newcommand{\qvec}[1]{\ensuremath{\widetilde{#1}}}
\newcommand{\qop}[2]{\ensuremath{{#1}\cdot{#2}}}
\newcommand{\qopseq}[2]{\ensuremath{{#1};{#2}}}

\newcommand{\qresultsind}[3]{\ensuremath{{#2}\overset{#1}{\leadsto}{#3}}}
\newcommand{\qresultsin}[2]{\qresultsind{}{#1}{#2}}
\newcommand{\qresultsindUp}[3]{\ensuremath{\begin{array}{l}%
             {#2}\overset{#1}{\leadsto}\\%
  \qquad{#3}}%
          \end{array}}
\newcommand{\qresultsinUp}[2]{\qresultsindUp{}{#1}{#2}}
\newcommand{\stackthree}[3]{\ensuremath{\begin{array}{l}{#1}\\{#2}\\{#3}\end{array}}}
\newcommand{\stackthreec}[3]{\ensuremath{\begin{array}{c}{#1}\\{#2}\\{#3}\end{array}}}
\newcommand{\stacktwo}[2]{\ensuremath{\begin{array}{l}{#1}\\{#2}\end{array}}}
\newcommand{\stacktwoc}[2]{\ensuremath{\begin{array}{c}{#1}\\{#2}\end{array}}}
\newcommand{\qmeasnob}[3]{{\begin{singlespace}\stackthree{\text{meas }{#1}:}{\quad \ket{0}=>#2}{\quad \ket{1}=>#3}\end{singlespace}}}
\newcommand{\qmeas}[3]{{\begin{singlespace}\left\{\stackthree{\text{meas }{#1}:}{\ \ket{0}=>#2}{\ \ket{1}=>#3}\right\}\end{singlespace}}}
\newcommand{\qcompose}[2]{\ensuremath{{#1};{#2}}}
\newcommand{\qtensor}[2]{\ensuremath{{#1};;{#2}}}
\newcommand{\qins}{\ensuremath{\mathscr{I}}}
\newcommand{\qmod}{\ensuremath{\mathscr{M}}}
\newcommand{\qcase}[4]{{\begin{singlespace}\left\{\stacktwo{\text{case }{#1}:}{\{{\ {#2}=>{#3}}\}_{#4}}\right\}\end{singlespace}}}
\newcommand{\quse}[2]{{\left\{\text{use }{#1}:\ \{{#2}\}\right\}}}
\newcommand{\tcls}{\ensuremath{\tau_C}}
\newcommand{\semins}[1]{\texttt{#1}}
\newcommand{\qifelse}[6]{{\begin{singlespace}\left\{\stackthree{\text{if }{#1}=>{#2}}{{\{\ {#3}=>{#4}}\}_{#5}}{\text{else }=>{#6}}\right\}\end{singlespace}}}


\newcommand{\qsmins}[1]{\texttt{#1}}

\newcommand{\qsminsparm}[1]{\ensuremath{#1}}
\newcommand{\qsminswithp}[2]{\texttt{#1}\qsminsparm{\ #2}}
\newcommand{\dmpelemqc}{\text{\texttt{Qc}}}
\newcommand{\qsmbool}[1]{\text{\texttt{#1}}}
\newcommand{\qsmfalse}{\qsmbool{False}}
\newcommand{\qsmtrue}{\qsmbool{True}}
\newcommand{\qstackMod}[1]{\texttt{#1}}

\newcommand{\terminalio}[1]{\texttt{#1}}

\newcommand{\nowiregate}[1]{*{\xy *+<.6em>{#1};p\save+LU;+RU **\dir{-}\restore\save+RU;+RD **\dir{-}\restore\save+RD;+LD **\dir{-}\restore\POS+LD;+LU **\dir{-}\endxy}}

\newcommand{\interpsem}[1]{\ensuremath{\left\llbracket {#1} \right\rrbracket}}

\newcommand{\Inflist}{\text{Inflist}}

\newcommand{\trspace}{\ensuremath{\qquad\qquad\qquad\qquad\qquad}}
\newcommand{\trbigspace}{\ensuremath{\qquad\qquad\qquad\qquad\qquad\qquad\qquad}}

\newcommand{\trspacefour}{\ensuremath{\qquad\qquad\qquad\qquad}}
\newcommand{\trspacethree}{\ensuremath{\qquad\qquad\qquad}}
\newcommand{\trspacetwo}{\ensuremath{\qquad\qquad}}

\newcommand{\ilsep}{\ensuremath{\blacktriangleright}}

\newcommand{\lqplmodifier}[1]{\text{\texttt{#1}}}
\newcommand{\IdOnly}{\lqplmodifier{IdOnly}}
\newcommand{\Left}{\lqplmodifier{Left}}
\newcommand{\Right}{\lqplmodifier{Right}}

\newcommand{\visctrl}[1]{\texttt{#1}}

\newenvironment{singlespace}{}{}