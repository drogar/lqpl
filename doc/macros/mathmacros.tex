\newcommand{\LG}{\ensuremath{L^1(G)}}

\newcommand{\supp}{\mathrm{supp}}
\newcommand{\KG}{\ensuremath{KG}}
\newcommand{\integers}{\ensuremath{ \mathbb{Z}}}
\newcommand{\naturals}{\ensuremath{ \mathbb{N}}}

\newcommand{\burnat}[1]{\ensuremath{ \mathbb{N}(#1)}}
\newcommand{\NA}{\burnat{A}}
\newcommand{\NNA}{\burnat{\NA}}
\newcommand{\NNNA}{\burnat{\NNA}}
\newcommand{\NB}{\burnat{B}}
\newcommand{\NNB}{\burnat{\NB}}
\newcommand{\NNNB}{\burnat{\NNB}}
\newcommand{\NC}{\burnat{C}}
\newcommand{\NNC}{\burnat{\NC}}
\newcommand{\NNNC}{\burnat{\NNC}}
\newcommand{\Nf}{\burnat{f}}
\newcommand{\Ng}{\burnat{g}}
\newcommand{\Nh}{\burnat{h}}
\newcommand{\NNf}{\burnat{\Nf}}
\newcommand{\NNg}{\burnat{\Ng}}
\newcommand{\NNh}{\burnat{\Nh}}
\newcommand{\N}{\naturals}
\newcommand{\add}[1]{\ensuremath{\mathsf{add}_{#1}}}
\newcommand{\adda}{\ensuremath{\mathsf{add}_A}}
\newcommand{\addna}{\ensuremath{\mathsf{add}_{\NA}}}
\newcommand{\addb}{\ensuremath{\mathsf{add}_B}}
\newcommand{\addnb}{\ensuremath{\mathsf{add}_{\NB}}}

\newcommand{\rec}[1]{\ensuremath{\mathbb{RC}(#1)}}
\newcommand{\recab}{\rec{A,B}}
\newcommand{\crs}[2]{\ensuremath{#1\times#2}}


\newcommand{\rationals}{\ensuremath{ \mathbb{Q}}}
\newcommand{\complexes}{\ensuremath{ \mathbb{C}}}
\newcommand{\reals}{\ensuremath{ \Re}}
\newcommand{\series}[3]{\ensuremath{#1_{#2}, \ldots, #1_{#3}}}
\newcommand{\pseries}[4]{\ensuremath{#1_{#3}:{#2}_{#3}, \ldots, #1_{#4}:{#2}_{#4}}}
\newcommand{\seriesd}[6]{\ensuremath{#1_{#2}, \ldots, #1_{#3}#4_{#5}, \ldots, #4_{#6}}}
\newcommand{\BigO}[1]{\ensuremath{\mathscr{O}({#1})}}
\newcommand{\etape}{\ensuremath{\sqcup\sqcup\ldots}}
\newcommand{\di}[1]{{\bfseries \itshape {#1}}}
\newcommand{\czero}{\ensuremath{\mathscr{C}_0}}
\newcommand{\cone}{\ensuremath{\mathscr{C}_1}}
\newcommand{\dzero}[1]{\ensuremath{D_0(#1)}}
\newcommand{\done}[1]{\ensuremath{D_1(#1)}}
\newcommand{\fab}{\ensuremath{f:A\to{}B}}
\newcommand{\gbc}{\ensuremath{g:B\to{}C}}
\newcommand{\hcd}{\ensuremath{h:C\to{}D}}

% for 613
\newcommand{\ora}[1]{\ensuremath{\overrightarrow{#1}}}
\newcommand{\ola}[1]{\ensuremath{\overleftarrow{#1}}}
\newcommand{\entails}[1]{\vdash^{#1}}
\newcommand{\nat}[1]{\{\text{succ}\colonn{#1},\text{zero}\colonn{()}\}}
\newcommand{\munat}{\mu{}x.\nat{x}}
\newcommand{\colonn}[1]{\coln#1}
%\newcommand{\colon}{\coln\!\!}
\newcommand{\coln}{\!\!:\!\!}
\newcommand{\dcolon}{\!\!::\!\!}
\newcommand{\banl}{\{\!|}
\newcommand{\banr}{|\!\}}
\newcommand{\fold}[2]{\ensuremath{\banl #1 | #2 \banr}}
%\newcommand{\implies}{\ensuremath{\Rightarrow}}
\renewcommand{\L}[1]{\ensuremath{\mathbb{L}(#1)}}
\newcommand{\LA}{\L{A}}

\newcommand{\Leaf}{\ensuremath{\text{\footnotesize\textsc{Leaf}}}}
\newcommand{\Node}{\ensuremath{\text{\footnotesize\textsc{Node}}}}
\newcommand{\clt}{\prgtransfunc{clt}}
\newcommand{\Zero}{\ensuremath{\text{\footnotesize\textsc{Zero}}}}
\newcommand{\Succ}{\ensuremath{\text{\footnotesize\textsc{Succ}}}}
\newcommand{\muzero}{\ensuremath{\ola{\Zero}}}
\newcommand{\musucc}{\ensuremath{\ola{\Succ}\cdot}}
\newcommand{\prgtransfunc}[1]{\ensuremath{\text{#1}}}
\newcommand{\zip}{\prgtransfunc{zip}}
\newcommand{\unzip}{\prgtransfunc{unzip}}
\newcommand{\zp}{\prgtransfunc{zp}}
\newcommand{\unz}{\prgtransfunc{unz}}
\newcommand{\nil}{\prgtransfunc{nil}}
\newcommand{\cons}{\prgtransfunc{cons}}
\newcommand{\tl}{\prgtransfunc{tl}}
\newcommand{\hd}{\prgtransfunc{hd}}
\newcommand{\rev}{\prgtransfunc{rev}}
\newcommand{\frev}{\prgtransfunc{frev}}
\newcommand{\acc}{\prgtransfunc{acc}}
\newcommand{\evens}{\prgtransfunc{evens}}
\newcommand{\evn}{\prgtransfunc{evn}}
\newcommand{\dbl}{\prgtransfunc{dbl}}
\newcommand{\get}{\prgtransfunc{get}}
\newcommand{\gtt}{\prgtransfunc{gtt}}
\newcommand{\gttc}[1]{\gtt\llbracket #1 \rrbracket}
\newcommand{\app}{\prgtransfunc{app}}
\newcommand{\appc}[1]{\app\llbracket #1 \rrbracket}
\newcommand{\munil}{\ensuremath{\ola{\nil}}}
\newcommand{\mucons}{\ensuremath{\ola{\cons}}}
\newcommand{\tlnu}{\ensuremath{\ora{\tl}\cdot}}
\newcommand{\hdnu}{\ensuremath{\ora{\hd}\cdot}}


\newcommand{\il}[1]{\ensuremath{\mathbb{IL}(#1)}}
\newcommand{\ila}{\il{A}}
\newcommand{\ilila}{\il{\ila}}
\newcommand{\ils}{\ensuremath{[a_1,a_2,a_3,\ldots]}}
\newcommand{\ilss}[1]{\ensuremath{a_{#1 1},a_{#1 2},a_{#1 3},\ldots]}}
\newcommand{\ilmataa}{\ensuremath{\begin{bmatrix} [a_{11},a_{12},a_{13},\ldots]\\
 [a_{21},a_{22},a_{23},\ldots]\\
 [a_{31},a_{32},a_{33},\ldots]\\
\vdots
\end{bmatrix}}}
\newcommand{\ilmataatr}{\ensuremath{\begin{bmatrix} [a_{11},a_{21},a_{31},\ldots]\\
 [a_{12},a_{22},a_{32},\ldots]\\
 [a_{13},a_{23},a_{33},\ldots]\\
\vdots
\end{bmatrix}}}



\newcommand{\nm}{\ensuremath{n\times{}m}}
\newcommand{\mn}{\ensuremath{m\times{}n}}

\newcommand{\specialcat}[1]{\textsc{#1}}
\newcommand{\ltrcat}[1]{\ensuremath{\mathfrak{#1}}}
\newcommand{\B}{\ltrcat{B}}
\newcommand{\C}{\ltrcat{C}}
\newcommand{\X}{\ltrcat{X}}
\newcommand{\sets}{\specialcat{Sets}}
\newcommand{\rel}{\specialcat{Rel}}
\newcommand{\mon}{\specialcat{Mon}}
\newcommand{\ring}{\specialcat{Rng}}
\newcommand{\cring}{\specialcat{CRng}}
\newcommand{\cat}{\textbf{\specialcat{Cat}}}
\newcommand{\poset}{\specialcat{Poset}}
\newcommand{\preorder}{\specialcat{Preorder}}


\newcommand{\obj}[1]{\ensuremath{#1_{obj}}}
\newcommand{\bottom}[1]{\perp_{#1}}
\newcommand{\finpower}{\mathscr{P}_{fin}}

\newcommand{\category}[4]{%
\begin{description}%
\item{\textbf{Objects: }}{#1}%
\item{\textbf{Maps: }}{#2}%
\item{\textbf{Identity: }}{#3}%
\item{\textbf{Composition: }}{#4}%
 \end{description}%
%
}

\newcommand{\pfcategory}[3]{%
\begin{description}%
\item{\textbf{Well-Defined: }}{#1}%
\item{\textbf{Identities: }}{#2}%
\item{\textbf{Associativity: }}{#3}%
 \end{description}%
%
}
%\newcommand{\categoryp}[4]{%
%\paragraph{Objects:}{#1}\\%
 %
%\paragraph{Maps:}{#2}\\%
 %
%\paragraph{Identity:}{#3}\\%
 %
%\paragraph{Composition:}{#4}\\%
 %
% }
\newenvironment{tageqnarray*}{\begin{equation*}\begin{array}{rclr}}{\end{array}\end{equation*}}

\newcounter{cdiagram}[section]
\renewcommand{\thecdiagram}{%
\thesection.\arabic{cdiagram}
}

\newcommand{\dicapt}[2]{%
{%
\refstepcounter{cdiagram}%
\begin{center}%
\thecdiagram\quad{}#1\label{#2} %
\end{center}%
}
}
\newcounter{annctr}
\newsavebox{\AnnpBox}
\newenvironment{annproof}
{\setcounter{annctr}{16}}
{\usebox{\AnnpBox}}

\newcommand{\annp}[1]{%
\stepcounter{annctr}%
\sbox{\AnnpBox}{\linebreak \theannctr{}.\quad{}#1}%
\theannctr%
}

\newcommand{\entailf}[1]{\vdash^{\annp{#1}}}


%\newcounter{cdiagram}[section]
%\newcommand{\dicapt}[2]{%
%\begin{center}%
%#1 #2 %
%\end{center}%


\newtheorem{theorem}{Theorem}[section]
\newtheorem{definition}[theorem]{Definition}
\newtheorem{notation}[theorem]{Notation}
\newtheorem{example}{Example}
\newtheorem{exercise}{Exercise}
\newtheorem{lemma}[theorem]{Lemma}
%\newtheorem*{cor}{Corollary}
%\newtheorem*{vhyp}{Valiant's Hypothesis}

\newcommand{\cclass}[2]{\ensuremath{\mathrm{#1}(#2)}}
\newcommand{\Time}[1]{\cclass{TIME}{#1}}
\newcommand{\Ntime}[1]{\cclass{NTIME}{#1}}
\newcommand{\Space}[1]{\cclass{SPACE}{#1}}
\newcommand{\Nspace}[1]{\cclass{NSPACE}{#1}}

% general math symbols
\newcommand{\union}{\ensuremath{\bigcup}}
\newcommand{\disjunion}{\ensuremath{\sqcup}}
\newcommand{\intersect}{\ensuremath{\bigcap}}
\newcommand{\logor}{\ensuremath{\lor}}
\newcommand{\logand}{\ensuremath{\land}}
\newcommand{\lognand}{\ensuremath{\barwedge}}
\newcommand{\natmap}{\ensuremath{\Rightarrow}}
\newcommand{\fctrmap}{\ensuremath{\to}}
\newcommand{\fnctrmap}{\fctrmap}
\newcommand{\fmap}{\ensuremath{\to}}
\newcommand{\produces}{\ensuremath{\to}}
\newcommand{\ladjoint}{\ensuremath{\dashv}}
\newcommand{\pproj}{\ensuremath{\preceq_p}}


\numberwithin{equation}{section}
\newenvironment{coproduct}{\left\{ \begin{array}{l}} {\end{array} \right\} }
%\newcommand{\coproductthree}[3]{\ensuremath{
%\left\{ \begin{array}{l} #1 \\ #2 \\#3 \end{array} \right\} }}
%\newcommand{\coproduct}[2]{\ensuremath{
%\left\{ \begin{array}{l} #1 \\ #2  \end{array} \right\} }}
\newcommand{\product}[2]{\ensuremath{
\left( \begin{array}{l} #1 \\ #2  \end{array} \right) }}
\newcommand{\inductf}[3]{\ensuremath{
\left[ \begin{array}{l}  \# #1\mapsto #2 (#3)  \end{array} \right]}}
\newcommand{\coinductf}[2]{\ensuremath{
\left[ \begin{array}{l}  \# #1\mapsto #2   \end{array} \right]}}

% adding commands for prooftheory
\newcommand{\context}[3] {%
\ensuremath{%
\begin{array}[b]{c}%
  \begin{array}[b]{|c|}%
  \hline %
   \begin{array}[b]{c}
     #2\quad%
   \end{array}\\
  \hline \\%
  #3\\%
  \hline%
  \end{array} \\%
#1 \\%
\end{array}%
}%
}

\newcommand{\muInfer}[4] {%
\ensuremath{%
\begin{array}[b]{c}%
  \begin{array}[b]{|c}%
  \hline %
   \begin{array}[b]{c|c}
     #2\quad & \quad#3%
   \end{array}\\
  \hline \\%
  #4\\%
  \hline%
  \end{array} \\%
#1 \\%
\end{array}%
}%
}


\newcommand{\muInferC}[5] {%
\ensuremath{%
\begin{array}[b]{c}%
  \begin{array}[b]{|c}%
  \hline %
   \begin{array}[b]{c|c|c}
     #2\quad & \quad#3&\quad#4%
   \end{array}\\
  \hline \\%
  #5\\%
  \hline%
  \end{array} \\%
#1 \\%
\end{array}%
}%
}

\newcommand{\nuInfer}[4] {
\ensuremath{
\begin{array}[b]{c}
  \begin{array}[b]{c|}
  \hline
   \begin{array}[b]{c|c}
     #2\quad & \quad#3%
   \end{array}\\
  \hline \\
  #4 \\
  \hline
  \end{array} \\
#1 \\
\end{array}
}
}


\newcommand{\nuInferC}[5] {
\ensuremath{
\begin{array}[b]{c}
  \begin{array}[b]{c|}
  \hline
   \begin{array}[b]{c|c|c}
     #2\quad & \quad#3&\quad#4%
   \end{array}\\
  \hline \\
  #5 \\
  \hline
  \end{array} \\
#1 \\
\end{array}
}
}

