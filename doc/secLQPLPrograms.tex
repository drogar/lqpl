\section{\lqpl{} programs}\label{sec:lqplprograms}
\lqpl{} programs  
consist of combinations of functions and data definitions, with 
one special function named \inlqpl{main}. 
The functions and data definitions are
\emph{simultaneously declared} and  so may be 
given in any order. The program will start executing at the \inlqpl{main}
 function. 

A physical program will typically consist of one or more source files
with the suffix \terminalio{.qpl}. Each source file may contain
functions and data definitions. It  may also
 \emph{import} the contents of other 
source files. The name of a source file is not significant in \lqpl. 
Common practice is to have one significant function per source file
and then to import all these files into the source file containing
the \inlqpl{main} function.

The above structure was chosen thinking of Haskell 
 \cite{peyton2003:haskell98} and C \cite{kernighan:c}. Global definitions 
of functions and types is borrowed from Haskell, while the \inlqpl{main} 
start point and import feature is a combination of Haskell and C.

\subsection{Data definitions}\label{subsec:datadefinitions}
\lqpl{} provides the facilities to define  datatypes with 
a syntax reminiscent of Haskell \cite{peyton2003:haskell98}. 

Natively, the language provides \inlqpl{Int}, 
\inlqpl{Qubit} and \inlqpl{Bool} 
types. \inlqpl{Bool} is the standard Boolean  type with values
\inlqpl{true} and \inlqpl{false}. \inlqpl{Int} is a standard 32-bit
integer. \inlqpl{Qubit} is a single \qubit{}.

In \lqpl{} both native types and other constructed datatypes may
be used in the definition of
constructed datatypes. These constructed datatypes may involve
 sums, products,
singleton types and parametrization of the constructed type.
For example, a type that is the sum of the integers and the Booleans
can be declared as  follows:
\begin{lstlisting}[style=linqplnonum]
        qdata Eitherib = {Left(Int) | Right(Bool)}
\end{lstlisting}
The above example illustrates the basic syntax of the 
data declaration. 
\paragraph{Syntax of datatype declarations.}
Each datatype declaration must begin with the keyword \inlqpl{qdata}. This is
followed by \emph{the type name}, which must be an identifier 
starting with an uppercase letter. The type name may also be followed by
 any number of \emph{type variables}. Type variables must be and identifier starting
with a lower case letter. Typically, a single letter is used.
This is then followed by an equals sign
and   completed by a list of \emph{constructors}. The list of constructors
must be surrounded by 
braces and each constructor must be separated from the others 
by a vertical bar. Each constructor is followed by an optional parenthesized
list of \emph{simple types}. 
Each simple type is either a built-in type, (one of 
\inlqpl{Int, Bool, Qubit}), a type variable that was used in the
type declaration, or another declared type, surrounded by 
parenthesis. \lqpl{} allows recursive
references to the type currently being declared. All constructors must
begin with an upper case letter. Constructors and types are in different
namespaces, so it is legal to have the same name for both. For example:
\begin{lstlisting}[style=linqplnonum]
       qdata Record a b = {Record(Int, a, b)}
\end{lstlisting}
In the above type definition,
 the first ``\inlqpl{Record}'' is the type, while the second is the 
constructor. The triplet ``\inlqpl{(Int,a,b)}'' is the product of the
type \inlqpl{Int} and the type variables \inlqpl{a} and \inlqpl{b}.
Since constructors may reference their own type and other 
declared types, recursive 
data types such as lists and various types of
 trees may be created:

\begin{lstlisting}[style=linqplnonum]
        qdata List a   = {Nil     | Cons(a, List(a))}
        qdata Tree a   = {Leaf(a) | Br(Tree(a), Tree(a))}
        qdata STree a  = {Tip     | Fork(STree(a), a, STree(a))}
        qdata Colour   = {Red     | Black}
        qdata RBSet a  = {Empty   |
	                  RBTip(Colour, RBSet(a), a, RBSet(a))}
        qdata RTree a  = {Rnode(a, List(a))}
        qdata Rose a   = {Rose(List(a, Rose(a)))} 
\end{lstlisting}
\subsection{Function definitions}\label{subsec:functiondefinitions}
Function definitions may appear in any order within a
\lqpl{} source file. 
\paragraph{Syntax of function definitions.} 
The first element of a function definition is
\emph{the name}, an identifier starting with a 
lower case letter. This is always followed by a double semi-colon and 
\emph{a signature}, which details the type and characteristics 
of input and output arguments. The final component of the 
function definition is
\emph{a body}, which is a block of \lqpl{} statements. 
Details of statements are given in \vref{sec:lqplstatements}.

The structure of function definitions is unique to \lqpl, although
broadly based on one of the acceptable syntaxes for C function definitions
as in \cite{kernighan:c}. The major difference occurs in the signature
which, as will be seen below, allows for both classical and quantum input
arguments and specifies the quantum outputs of a function. Another difference
is that \lqpl{} defines all functions globally, hence there is no
requirement for declarations separate from the definitions.

Let us examine two examples of functions. The first, in \vref{fig:defsec:gcd}
 is a fairly 
standard function to determine the greatest common divisor of two 
integers.

\begin{figure}[htbp]
\begin{singlespace}
\lstinputlisting[style=linqpl]{examplecode/gcd.qpl}
\end{singlespace}
\caption{\lqpl{} function to compute the GCD}\label{fig:defsec:gcd}
\end{figure}

The first line has  the name of the function, \inlqpl{gcd}, a separating double colon, and
 the signature. 
The signature of the function is  \inlqpl{(a:Int, b:Int | ; ans:Int)}.
This signature tells the compiler that \inlqpl{gcd} expects
two input arguments, each of type \inlqpl{Int} and that they are
 \emph{classical}. The compiler deduces this
from the fact that they both appear before the '\inlqpl{|}'.
In this case there are no  quantum input arguments
as there are no parameters between the '\inlqpl{|}' and the
'\inlqpl{;}'. The last parameter tells us that this function returns
one quantum item of type \inlqpl{Int}. Returned items are always  
quantum data.

The signature specifies variable names for the parameters
 used in the body. All input parameters are available as variable
names in expression and statements. Output parameters are available to be
assigned and, indeed, must be assigned by the end of the function.


The next example, in \vref{fig:defsec:inttoqubit}
 highlights the linearity of variables in \lqpl. This function is 
 used to create a list of \qubit{}s corresponding to the
\bit{} representation of an input integer.

At line 1,  the program uses the import command.
\inlqpl{#Import} must have a file name directly after it. This 
command directs the compiler to stop reading from the current file
and to read  code in the imported file until the end of that
file, after which it  continues with the current 
file.  The compiler will not reread the same file in a single
 compilation, and it will import from any file.


\begin{figure}[htbp]
\begin{singlespace}
\lstinputlisting[style=linqpl]{examplecode/intToQubitList.qpl}
\end{singlespace}
\caption{\lqpl{} function to create a \qubit{} register}
\label{fig:defsec:inttoqubit}
\end{figure}

For example, consider a case with three source files, A, B and C.
Suppose file A has import commands for both B and  C, with B being
imported first. Further suppose that  file B 
 imports C. The compiler will start reading A, suspend at the first 
import and start reading B. When it reaches B's \inlqpl{#Import} of C, it
will suspend the processing of B and read C. 
After completing the read of C, the compiler
reverts to processing B.
After completing the read of B, the compiler does a 
final reversion and finishes  processing  A. 
However, when A's \inlqpl{#Import} of C
is reached, the compiler will ignore this import
 as it keeps track of the fact C has
already been read.

In the signature on lines 
\ref{line:itqlfunctionstart}-\ref{line:itqlfunctionend},  
the function accepts
one quantum parameter of type \inlqpl{Int}. It returns
a  \inlqpl{Qubit} list and an \inlqpl{Int}. The integer returned, in
this program, is computed to have 
the same value as the one passed in. If this had not been
specified in this way, 
\emph{the integer would have been destroyed by the function}.  Generally, any usage
of a quantum variable destroys that variable. 

In the body of the function, note the \inlqpl{use n} in the
 block of statements. This allows  repeated 
use of the variable \inlqpl{n} at lines \ref{line:itqln0}, \ref{line:itqln1},
\ref{line:itqln2}, \ref{line:itqln3} and \ref{line:itqln4}. In these uses,
\inlqpl{n} is a classical variable, no longer on the quantum stack. The last
usage on line \ref{line:itqln4} where \inlqpl{n} is assigned to itself, 
returns \inlqpl{n} to the quantum world.



