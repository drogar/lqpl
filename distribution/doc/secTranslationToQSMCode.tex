\section{Translation of \lqpl{} to stack machine code}\label{sec:translationtoqsmcode}
This section will discuss the code produced by the 
various statements and expressions in an \lqpl{} program.
An \lqpl{} program consists of
a collection of data definitions and procedures. Data definitions do not
generate any direct code but do affect the code generation of statements
and expressions.

Each procedure will generate code. A procedure consists of a collection of
statements each of which will generate code. Some statements may have other
statements of expressions as dependent pieces, which again will generate 
code.

The code generation in the compiler, and the description here, follows a
standard recursive descent method.
\subsection{Code generation of procedures}\label{subsec:cgprocedures}

The code generated for each procedure follows a standard pattern of: 
procedure entry; procedure statements; procedure exit. The procedure statements 
portion is the code generated for the list of statements of the procedure, each of
which is detailed in \vref{subsec:cgstatements}.

\subsubsection{Procedure entry}
Each procedure is identified in QSM by an entry point, using an assembler directive.
This directive is a mangled name of the procedure, followed by the keyword 
\qsmins{Start}. The only exception to this is the special procedure 
\qsmins{main} which is generated without mangling. \qsmins{main} is always
the starting entry point for a QSM program.

\subsubsection{Procedure exit}
The end of all procedures is denoted by another assembler directive, 
\qsmins{EndProc}. For all procedures except \qsmins{main}, the code generation
determines how many classical variables are being returned by the procedure
and emits a \qsmins{Return} $n$ instruction, where $n$ is that count\footnote{This
functionality is currently not available in \lqpl{}, but may be re-introduced at
a later date.}.

\subsubsection{Procedure body}
The code for each statement in the list of statements is generated and used as the 
body of the procedure.
\begin{figure}[htbp]
\centering
\subfloat[Coin flip code]{
\begin{singlespace}
\lstinputlisting[style=linqpl]{examplecode/coinnolbls.qpl}
\end{singlespace}}\qquad
\subfloat[Generated  code]{
\begin{singlespace}
\lstinputlisting[style=linqpl]{examplecode/coin.generated}
%VerbatimInput[numbers=left,numbersep=3pt]{examplecode/coin.generated}
\end{singlespace}
}
\caption{\lqpl{} and QSM coin flip programs}\label{fig:cg:coin}
\end{figure}
As an example, see the coin flip code and the corresponding generated QSM code
in \vref{fig:cg:coin}.

\subsection{Code generation of statements}\label{subsec:cgstatements}
Each statement in \lqpl{} generates code. The details of the
code generation for each statement are given in the following pages, together
with examples of actual generated code.
\subsubsection{Assignment statements}
The sub-section describes code generation for quantum assignment statements and
assignments to variables on the classical stack. The classical assignment ($:=$)
statement is described with the \inlqpl{use} statement below, as it is
syntactic sugar for that statement.

An assignment of the form $i = \langle expr \rangle$ is actually 
broken down into 5 special cases. The first is when the left hand side is
an in-scope variable that is on the classical stack. The other four all 
deal with the case of a quantum variable, which is either introduced or
overwritten. The four cases depend on the type of expression on the 
right hand side. Each paragraph  below will identify which case is
being considered and then describe the code generation for that case.

\paragraph{Left hand side is a classical variable.} In this case, generate
the code for the expression on the right hand side (which will be classical
in nature). This leaves the expression value at the top of the classical
stack. Now, emit a \qsmins{CPut} instruction which will copy that value
into the location of the classical variable.

\begin{center}
\begin{tabular}{p{1in}p{.5in}p{1.5in}}
{\begin{singlespace}
\begin{lstlisting}[style=linqpl]
i = 5;
\end{lstlisting}
\end{singlespace}}
 & { \quad \quad \raisebox{.8em}{$\implies$}} &
{\begin{singlespace}
\begin{lstlisting}[style=linqpl]
     CLoad 5
     CPut -2
\end{lstlisting}
\end{singlespace}}
\end{tabular}
\end{center}


\paragraph{Right hand side is a classical expression.} First, generate the
expression code, which leaves the value on the top of the classical stack.
Then emit a \qsmins{QMove} instruction with the name of the left hand side.
This will create a new classical node, which will be set to the value of the
top of the classical stack.


\begin{center}
\begin{tabular}{p{1in}p{.5in}p{1.5in}}
{\begin{singlespace}
\begin{lstlisting}[style=linqpl]
i = 5;
\end{lstlisting}
\end{singlespace}}
 & { \quad \quad \raisebox{.8em}{$\implies$}} &
{\begin{singlespace}
\begin{lstlisting}[style=linqpl]
     CLoad 5
     QMove i
\end{lstlisting}
\end{singlespace}}
\end{tabular}
\end{center}


\paragraph{Right hand side is a constant \qubit.} Emit the 
\qsmins{QLoad} instruction  with the \qubit{} value and the left hand side
variable name.

\begin{center}
\begin{tabular}{p{1in}p{.4in}p{1.5in}}
{\begin{singlespace}
\begin{lstlisting}[style=linqpl]
q = |1>;
\end{lstlisting}
\end{singlespace}}
 & { \raisebox{-1.3em}{$\implies$}} &
{\begin{singlespace}
\begin{lstlisting}[style=linqpl]
     QLoad q |1>
\end{lstlisting}
\end{singlespace}}
\end{tabular}
\end{center}


\paragraph{Right hand side is an expression call.}
First, emit the code for the expression call. This will leave the
result quantum value on the top of the quantum stack. If the
formal name given by the procedure definition is the same as the
left hand side name, do nothing else, as the variable is already
created with the proper name. If not, emit a \qsmins{QName} instruction
to rename the last formal parameter name to the left hand side name.

\begin{center}
\begin{tabular}{p{2in}p{.3in}p{1.75in}}
{\begin{singlespace}
\begin{lstlisting}[style=linqpl]
random :: (maxval :Int; 
       rand :Int) = {...}
...
x = random(15);
\end{lstlisting}
\end{singlespace}}
 & { \qquad \qquad \quad \quad \qquad \qquad $\implies$} &
{\begin{singlespace}
\begin{lstlisting}[style=linqpl]
CLoad 15
QMove c18
QName c18 maxval //In
Call 0 random_fcdlbl0
QName rand x //Out
\end{lstlisting}
\end{singlespace}}
\end{tabular}
\end{center}


\paragraph{Right hand side is some other expression.}
Generate the code for the expression. Check the name on the top of
the stack. If it is the same as the left hand side name, do nothing else,
otherwise emit a \qsmins{QName} instruction.


\begin{center}
\begin{tabular}{p{2in}p{.3in}p{1.5in}}
{\begin{singlespace}
\begin{lstlisting}[style=linqpl]
outqs = Cons(q, inqs');
\end{lstlisting}
\end{singlespace}}
 & { \qquad \qquad \quad \quad \qquad \qquad $\implies$} &
{\begin{singlespace}
\begin{lstlisting}[style=linqpl]
    QCons c4 #Cons
    QBind inqs'
    QBind q
    QName c4 outqs
\end{lstlisting}
\end{singlespace}}
\end{tabular}
\end{center}


\subsubsection{Measurement code generation}
%FIXME
% We *should* be able to get away with, yes, m0, m1 and mf, but no spurious
% EndQC(now swap) statement.
Measurement will always have two subordinate sets of statements,
respectively for the \ket{0} and \ket{1} cases. 
The generation for the actual statement will handle the requisite branching.

The code generation first acquires three new labels, $m_0,m_1$ and $m_f$. It
will then emit a \qsmins{Measure}~$m_0\ m_1$ statement, followed by 
a \qsmins{Jump}~$m_f$. Recall from the transitions 
in \vref{subsec:transitiondiagrams} that when the machine executes
the \qsmins{Measure} instruction, \emph{it then generates and executes a}
\qsmins{SwapD} instruction. The \qsmins{Jump} will be executed when 
all branches of the \qubit{} have been executed.

Then, for each of the two sub blocks ($i\in\{0,1\}$), 
I emit a \qsmins{Discard} labelled with $m_i$. This is followed by the 
code generated from the corresponding block of statements. Finally a 
\qsmins{SwapD} is emitted.

The last instruction generated is a \qsmins{NoOp} which is labelled
with $m_f$.


\begin{center}
\begin{tabular}{p{2in}p{.3in}p{1.5in}}
{\begin{singlespace}
\begin{lstlisting}[style=linqpl]
measure q of 
 |0> => {n1 = 0}
 |1> => {n1 = 1};
\end{lstlisting}
\end{singlespace}}
 & { \qquad \qquad \quad \quad \qquad \qquad $\implies$} &
{\begin{singlespace}
\begin{lstlisting}[style=linqpl]
   QPullup q
   Measure l7 l8
   Jump l9
l7 QDiscard
   CLoad 0
   QMove n1
   SwapD
l8 QDiscard
   CLoad 1
   QMove n1
   SwapD 
l9 NoOp
\end{lstlisting}
\end{singlespace}}
\end{tabular}
\end{center}


\subsubsection{Case statement code generation}

Case statement generation is conceptually similar to that of
measurement. The  differences are primarily due to the variable 
number of case clauses and the need to instantiate the variables of the 
patterns on the case clauses.

As before, the expression will have its code generated first.
Then, the compiler will use a function to return a list of triples of
a constructor, its generated label and the corresponding code for
 each of the case clauses. 
The code generation done by that function is detailed below.

At this point, the code generation resembles measurement generation.
The compiler generates a label $c_f$ and emits
a \qsmins{Split} with a list of constructor / code label pairs which
have been returned by 
the case clause generation. This is followed by emitting
a \qsmins{Jump}~$c_f$. 
After the \qsmins{Jump} has been emitted, 
the code generated by the case clause generation is emitted.

The final instruction generated is a \qsmins{NoOp} which is labelled
with $c_f$.

\paragraph{Case clause code generation.} 
The code generation for a case clause has a rather complex prologue which
ensures the assignment of any bound variables to the patterns in the 
clause.

First, the code generator gets
 a new label $c_l$ and then calculates the unbinding code as follows:
For each \emph{don't care} pattern in the clause, code is 
generated to delete the corresponding bound node. 
This is done by first getting a new stack name $nm$, then
adding the instructions \qsmins{QUnbind}~$nm$, \qsmins{Pullup}~$nm$ and
\qsmins{QDelete}~$nm$. This accomplishes the unbinding of that node and
removes it from the quantum stack. Note that \qsmins{QDelete} is used here
to ensure that all subordinate nodes are removed and that no spurious data
is added to the classical stack in the case of the don't care node being a
classical value.

For each  named pattern $p$, only the
instruction \qsmins{QUnbind}~$p$ is added.

The program now has a list of instructions that will accomplish unbinding
of the variables. Note this list may be empty, e.g., the \inlqpl{Nil}
constructor for \inlqpl{List}.

The clause generation now creates its own return list. When the 
unbinding list is not empty, these instructions are added first, with
the first one of them being labelled with $c_l$.This is followed by
a \qsmins{Discard} which discards the decomposed data node. 
When the unbind list  is
empty, the first instruction is the \qsmins{Discard} labelled by $c_l$.

Then, the code generated by the statements in the case clause block are
added to the list. Finally, a \qsmins{SwapD} is added at the end.

The lists from all the case clauses are combined and this is returned to the
case code generation.

The example at the end of the use clause will illustrate both the case clause 
and the use statement code generation.

\subsubsection{Use and classical assign code generation}
As described earlier in \vref{subsec:usestatements}, the classical assignment, 
\inlqpl{v := exp} is syntactic sugar for a variable assignment followed by
a \inlqpl{use} statement. The code generation for each is 
handled the same way.

There are two different cases to consider when generating this code.
 The \inlqpl{use} statement
may or may not have subordinate statements. In the case where it does not have
any subordinate statements, (a classical assign or a
 use with no block), the scope
of the classical variables in the use 
extends to the end of the enclosing block.

The case  of a use with a subordinate block  is presented first.

\paragraph{Use with subordinate block code generation.}
While similar to the generation for the measure and case statements,
there are differences. The two main differences are that there is only one
subordinate body of statements and that  multiple variables may be
used.

In the case where there is a single use variable $nu$, the generator gets
the body label $u_b$ and end label $u_e$. It then emits a 
\qsmins{Pullup}~$nu$, a \qsmins{Use}~$u_b$ and a \qsmins{Jump}~$u_e$.
This is followed by emitting
 a \qsmins{Discard} labelled by $u_b$ and the code generated
by the subordinate body of statements and a \qsmins{SwapD} instruction.
This is terminated by a \qsmins{NoOp} labelled by $u_e$.

When there are multiple names $n_1,n_2,\ldots,n_j$, the generator first
recursively generates code assuming the same body of 
statements but with a use statement that only has the variables 
$n_2,\ldots,n_j$. This is then used as the body of code for
a use statement with only one variable $n_1$, which is generated
in the same manner as in the above paragraph.

\paragraph{Use with no subordinate block.}
To properly generate the code for this, including the \qsmins{SwapD} and
end label, the generator uses
 the concept of delayed code. The prologue (\qsmins{Use},
\qsmins{Jump} and \qsmins{Discard}) and epilogue (\qsmins{SwapD}, \qsmins{NoOp})
are created in the same manner as the use with the subordinate block.
The prologue is emitted at the time of its generation. The epilogue, however,
is added to a push-down stack of delayed code which is emitted at the
end of a block.  See the description of the block code generation for more
details on this. These items are illustrated in \vref{fig:gencasecode}.

\begin{figure}[htbp]
\begin{center}
\begin{tabular}{p{2in}p{.3in}p{2.5in}}
{\begin{singlespace}
\begin{lstlisting}[style=linqpl]
case l of
  Nil => {i = 0}
  Cons (_, l1) => {
    n= len(l1);
    use n;
    i = 1 + n;
}
\end{lstlisting}
\end{singlespace}}
 & { \qquad \qquad \quad \quad \qquad \qquad $\implies$} &
{\begin{singlespace}
\begin{lstlisting}[style=linqpl]
 Split (#Nil,l0) (#Cons,l1) 
   Jump l4
l0 QDiscard
   CLoad 0
   QMove i
   SwapD
l1 QUnbind c0
   QUnbind l1
   QDiscard
   QPullup c0
   QDiscard 
   QPullup l1
   QName l1 l
   Call 0 len_fcdl0
   QName i n
   QPullup n
   Use l2
   Jump l3
l2 QDiscard 
   CLoad 1
   CGet 0
   CApply +
   QMove i
   SwapD //For Use
l3 NoOp
    SwapD //For Split
l4 NoOp
\end{lstlisting}
\end{singlespace}}
\end{tabular}
\end{center}
\caption{QSM code generated for a case and use statement.}\label{fig:gencasecode}
\end{figure}



\subsubsection{Conditional statements}
The \inlqpl{if ... else} statement allows the programmer to specify an
unlimited number of classical expressions to control blocks of code. Typically,
this is done within a \inlqpl{use} statement based upon the variables used.

The statement code generation is done by first requesting a new label, $g_e$.
This label is used in the next step,  generating the 
code for all of the guard clauses. The generation is completed by emitting a 
\qsmins{NoOp} instruction labelled by $g_e$.

\paragraph{Guard clauses.}\label{para:guardclauses} 
Each guard clause consists of an expression and
list of statements. The code generator first emits
 the code to evaluate the expression. 
Then, a new label $g_l$ is requested and the instruction \qsmins{CondJump}~$g_l$
is emitted. The subordinate statements are generated and emitted, 
considering them as a 
single block. The concluding instruction is a \qsmins{Jump}~$g_e$.

At this point, if there are no more guard clauses, a \qsmins{NoOp} instruction,
labelled by $g_l$ is emitted, otherwise the code generated by the remaining
guard clauses is labelled by $g_l$ and emitted.
%FIXME - check that this is right and what really happens - appears that
% multiple branches would not be right...

\begin{center}
\begin{tabular}{p{2in}p{.3in}p{2.5in}}
{\begin{singlespace}
\begin{lstlisting}[style=linqpl]
if b == 0 => { 
   theGcd = a;
} else =>  { 
   (theGcd) = 
     gcd(b, a mod b);
}
\end{lstlisting}
\end{singlespace}}
 & { \qquad \qquad \quad \quad \qquad \qquad $\implies$} &
{\begin{singlespace}
\begin{lstlisting}[style=linqpl]
    CGet -2
    CLoad 0
    CApply ==
    CondJump lbl2
    CGet -1
    QMove theGcd
    Jump lbl1
lbl2 CLoad True //else
    CondJump lbl3
    CGet -1
    CGet -2
    CApply %
    QMove c0
    CGet -2
    QMove c1
    QName c1 a
    QName c0 b
    Call 0 gcd_fcdlbl0
    Jump lbl1
lbl3 NoOp
lbl1 NoOp
\end{lstlisting}
\end{singlespace}}
\end{tabular}
\end{center}


\subsubsection{Function calling and unitary transforms}\label{subsubsec:cgfunctioncalls}
The code generation of these two statements is practically the same, with
the only difference being that built in transformations have a special
instruction in QSM, while executing a defined function requires
the \qsmins{Call} instruction.

In each case, the statement allows for input classical and quantum expressions
and output quantum identifiers.

The first step is the generation of the code for the input 
classical expressions. These are generated in reverse order so that the 
first parameter is on the top of the classical stack, the second is next
and so forth. Then, the input quantum expressions are generated,
with names of these expressions being saved. Note that it is possible to use
expressions which are innately classical (e.g., constants and variables on
the classical stack) as a quantum expression. The compiler will generate 
the code needed to lift it to a quantum expression. See 
\vref{subsec:cglifting} for the details.

At this stage, the two
types differ slightly. For the unitary transformation case, the
code ``\qsmins{QApply}~$n$~!$t$'' is emitted, where $n$ is the number of 
classical arguments and $t$ is the name of the built in transform. The
exclamation mark is part of the QSM assembler syntax for transform names.
In a defined function, renames of the input quantum expressions to
the names of the input formal ids are generated, 
by emitting a series of \qsmins{QName} 
instructions. This is followed by emitting a \qsmins{Call}~$n$~$f$, 
where $n$ is again the number of classical arguments and $f$ is the 
internally generated name of the function.

In both cases, the code checks the formal return parameter names against
the list of return value names. For each one that is different, a 
\qsmins{QName}~$frml$~$retnm$ is emitted.

See the previous code list between the  \qsmins{CondJump lbl3} and
\qsmins{lbl3 NoOp} for an example of this.



\subsection{Code generation of expressions}\label{subsec:cgexpression}
In the previous sub-section, a number of examples of code generation
were given. These also illustrated most of the different aspects of 
expression code generation. A few additional examples are given below.

\subsubsection{Generation of constants}
There are three possible types of constants, a Boolean, an integer or 
a \qubit. Note that constructors are considered a different class of 
expression.

For both Boolean and integers, the compiler emits a \qsmins{CLoad}~$val$ 
instruction. For a \qubit, it creates a new name $q$ and emits
a \qsmins{QLoad}~$q$~$qbv$ instruction.

Examples of these may be seen in the sub-sub-section on assignments.

\subsubsection{Generation of classical arithmetic and Boolean expressions}
In all cases, these types of expressions are calculated solely on 
the classical stack. Whenever the generator encounters 
 an expression of the form
\[ e_1\ op\ e_2\]
it first emits the code to generate $e_2$, followed by the code to generate
$e_1$. This will leave the result of $e_1$ on top of the stack with $e_2$'s
value right below it. 
It then emits the instruction \qsmins{CApply}~$op$, which will
apply the operation to the two top elements, replacing them with the
result.

The Boolean \emph{not} operation is the only operation of arity 1. 
Code generation is done in the same manner. The generator emits code for the 
expression first, followed by \qsmins{CApply}~$\neg$.

 See  under Guard clauses, \vref{para:guardclauses} for
examples.

\subsubsection{Generation of variables}
The semantic analysis of the program will split this into two cases; 
classical variables and quantum variables. Each are handled differently.

\paragraph{Classical variables.} These variables are on the classical
stack at a specific offset. The use of them in an expression means 
that they are to be \emph{copied} to the top of the classical stack.
The code emitted is \qsmins{CGet}~$offset$, where $offset$ is the 
offset of the variable in the classical stack.

\paragraph{Quantum variables.} In this case, the variables are at a
specific address of the quantum stack. These variables are not allowed
to be copied, so the effect of this code is to rotate the quantum stack 
until the desired variable is at the top. The other consideration
is that these variables have a linearity implicitly defined in their
usage. In the compiler, this is handled by the semantic analysis phase,
but the code generation needs to also consider this. The compiler will
add this variable to a \emph{delayed deletion} list. After completion
of the statement with this variable expression, the variable will be 
deleted unless:
\begin{itemize}
\item{} The statement has deleted it. (Measure of a \qubit, for example, will 
directly generate the deletion code.)
\item{} It is recreated as the result of an assignment, function call or
transformation.
\end{itemize}

\subsubsection{Code generation for expression calls}
Each expression call is generated in substantially the same way 
as the code for
a call statement as in \ref{subsubsec:cgfunctioncalls}. The only 
difference is that the name of the final return variable will not be known
and is therefore set to the same name as the name of the last output formal
parameter. As an example, consider:

\begin{center}
\begin{tabular}{p{4in}}
{\begin{singlespace}
\begin{lstlisting}[style=linqpl]
gcd::(a:Int, b:Int; theGcd:Int)= 
{ use a,b in 
  { if b == 0 => { theGcd   = a} 
    else      => { (theGcd) = gcd(b, a mod b)}
  }   
}
\end{lstlisting}
\end{singlespace}}
\end{tabular}
\end{center}


Suppose this is defined in a program, and at some point, it is called
as an expression in the program :
\inlqpl{gcd(5,n)}. The code generated for this expression will
then leave the integer node named \emph{theGcd} on the top
of the quantum stack.

\subsubsection{Generation of constructor expressions}
These expressions are used to create new data type nodes, such as lists, trees
etc. Constructors are similar to functions in that they expect 
an expression list as input and will return a new quantum variable of a
specific type. In \lqpl{} they are somewhat simpler as the input expressions
are all expected to be quantum and there is a single input only. Just as in
function calls, any classical expressions input to the constructor
will be lifted to a quantum expression.

The first step is for the compiler to emit code that will evaluate and lift
any of the input expressions. The names of each of these expressions is 
saved. Then, it creates a new name $d_c$ and emits the
code \qsmins{QCons}~$d_c$~$\#cid$.

The final stage is to emit a \qsmins{QBind}~$nm_{e_i}$ for each of the
input expression, \emph{in reverse order} to what was input. The next  example 
illustrates this.

\begin{center}
\begin{tabular}{p{2in}p{.3in}p{2.7in}}
{\begin{singlespace}
\begin{lstlisting}[style=linqpl]
lt = Cons(|0>, 
       Cons(|0>,
       iTZQList(2*n)));
\end{lstlisting}
\end{singlespace}}
 & { \qquad \qquad \quad \quad \qquad \qquad $\implies$} &
{\begin{singlespace}
\begin{lstlisting}[style=linqpl]
    QLoad c22 |0> 
    QLoad c23 |0>
    CLoad 2  //for the call
    CGet -1
    CApply *
    QMove c24
    QName c24 n
    Call 0 iTZQList_fcdlbl5
    QCons c25 #Cons
    QBind nq 
    QBind c23 //c23:nq
    QCons c26 #Cons
    QBind c25
    QBind c22 //c22:c23:nq
    QName c26 lt
\end{lstlisting}
\end{singlespace}}
\end{tabular}
\end{center}



\subsection{Lifting of classical expressions to quantum expressions.}\label{subsec:cglifting}
When the compiler requires a quantum expression, but has been given a
classical one, it first generates the classical expression. This leaves
the expression value on top of the classical stack. The compiler will now
generate a new unique name $l_c$ and emit the instruction 
\qsmins{QMove}~$l_c$. This now moves the value from the classical stack
to the quantum stack.
