\lstset{style=inlinqpl}
\section{Introduction to \lqpl}\label{sec:introlqpl}
This chapter  presents an overview of the linear quantum
programming language. Explanations of \lqpl{} programs, statements, 
and expressions are given.


\lqpl{} is  a language for experimenting with  quantum
algorithms. The language provides an expressive syntax for creating functions 
and defining and working
 with different datatypes. \lqpl{} has \qubit{}s as first class
citizens of the language, together with quantum control. 
Classical operations and classical control are also available to 
work with classical data.

The language design started from QPL in \cite{selinger04:qpl}
and rapidly evolved to a point where a direct comparison is 
somewhat difficult.
Major differences are the type system, the syntax and structure of functions
and the choices of individual statements.

\subsection{Functional versus imperative}\label{subsec:lqplfunctionvsimperative}
\lqpl{} is a functional language that uses single assignment. It resembles
QPL in this aspect rather than QML of \cite{alti05:functionalQMLlics}. 
Single assignment means that a variable always has a unique value until its
use. (See \ref{subsec:lqpllinearity} below.)

Like most functional languages, side effects are not allowed in functions, 
\emph{other than those that occur due to quantum entanglement}. Side effects in imperative languages
are those where a global variable is updated or values are read or written. 
An example of this in an imperative
 quantum language is a procedure in QCL from 
\cite{omer:2000}. \lqpl{} does not have the concept of a global variable. 
Currently, I/O is undefined.  Side effects from quantum entanglement can 
occur when a \qubit{} is passed as a parameter to a function and it is
operated on in the function.

\subsection{Linearity of \lqpl}\label{subsec:lqpllinearity}
The language \lqpl{} treats all quantum variables as \emph{linear}. This 
means that any variable \emph{may only be used once}. 
The primary reason for implementing this is the underlying aspect of 
linearity of quantum systems, as exemplified by 
 the \emph{no-duplication} rule which must be respected at all times. 
This allows us to provide
compile-time checking that enforces this rule.

The compiler and language do
 provide ways to ``ease the burden'' of linear thinking. 
For example, function calls (see \vref{subsec:functioncalls}) 
provide a specialized syntax for variables
which are both input and output to a function. The ability to use
a classical value (integer or boolean) multiple times is handled by  
\inlqpl{use} statements (see \vref{subsec:usestatements}), which place values on to the
classical stack where the values may be used multiple times.

\begin{figure}[htbp]
\lstinputlisting[style=linqpl]{examplecode/LengthList.qpl}
\caption{\lqpl{} code to return the length of the list}\label{fig:lenSQPL}
\end{figure}

An example illustrating linearity is given in \vref{fig:lenSQPL}. In
line \ref{line:lengthlist:len}, the function \inlqpl{len} is defined
as taking one argument of type \inlqpl{List(a)} and returning
a variable of type \inlqpl{Int}. The input only argument, \inlqpl{listIn},
\emph{must be destroyed in the function}. When the
case statement refers to \inlqpl{listIn}, the argument is 
destroyed, fulfilling the requirement of the function to do so.



