\section{Basic examples}\label{sec:examplesBasic}
This section covers a number of the standard examples of 
quantum algorithms covered in most introductions to the subject.
\subsection{Quantum teleportation function}\label{subsubsec:quantumTeleportationExample}
The \lqpl{} program shown in \vref{fig:stackTeleportation}
 is an implementation of a function that will accomplish 
quantum teleportation as per the circuit shown  in 
\vref{qc:quantumTeleportation}. (See also \cite{watrous:lecnotes} or
\cite{neilsen2000:QuantumComputationAndInfo}). It also provides a separate function
to place two \qubits{} into the EPR state.

\begin{figure}[htbp]
\centerline{%
\Qcircuit @C=1em @R=.7em {
\lstick{\ket{\nu}} & \ctrl{1} & \gate{H} & \measureD{M_1} & \cw &  \control \cw  \cwx[2] \\
\lstick{\mbox{A}}  & \targ & \qw & \measureD{M_2} & \control \cw  \cwx[1] \\
\lstick{\mbox{B}} & \qw &\qw & \qw & \gate{X} & \gate{Z} & \qw & \rstick{\ket{\nu}}\\
& \rstick{\ket{s_1}}& \rstick{\ket{s_2}} & & \lstick{\ket{s_3}} & &
\lstick{\ket{s_4}} }}
\caption{Quantum teleportation}
\label{qc:quantumTeleportation}
\end{figure}

Note that the teleport function, similarly to the circuit,
 does not check the precondition that \qubit{}s $a$ and $b$ are in the
EPR state, which is required to actually have teleportation work.

\begin{figure}[htbp]
\lstinputlisting[style=linqpl]{examplecode/teleport.qpl}
\caption{\lqpl{} code for a teleport routine}\label{fig:stackTeleportation}
\end{figure}

\subsection{Quantum Fourier transform}
The \lqpl{} program to implement the quantum Fourier transform in 
\vref{fig:qft} uses
two recursive routines, \inlqpl{qft} and \inlqpl{rotate}. These
functions assume the \qubits{} to transform are in a \inlqpl{List}.
(See also \cite{selinger04:qpl} and either \cite{watrous:lecnotes} or
\cite{neilsen2000:QuantumComputationAndInfo}).

The routine \inlqpl{qft} first applies the \Had{} transform to the
\qubit{} at the head of the list, then uses the \inlqpl{rotate} routine
to recursively apply the correct $Rot$ transforms controlled by the
other \qubits{} in the list. \inlqpl{qft} then recursively calls itself on
the remaining \qubits{} in the list.

\begin{figure}[htbp]
\lstinputlisting[style=linqpl]{examplecode/qft.qpl}
\caption{\lqpl{} code for a quantum Fourier transform}\label{fig:qft}
\end{figure}

\subsection{Deutsch-Jozsa algorithm}\label{appsubsec:djalgorithm}
The \lqpl{} program to implement the Deutsch-Jozsa algorithm is in 
\vref{fig:dj} with supporting routines in \ref{fig:initlist}, 
\ref{fig:addnzerps}, \ref{fig:measureinps}. The 
\inlqpl{hadList} function is defined in \vref{fig:oflistutils}.

The algorithm decides if a function is balanced or constant on 
$n$ \bits. This implementation requires supplying the number of
\bits{} / \qubits{} used by the function, so that the input can 
be prepared. Additionally, it currently requires hand-writing the 
\emph{oracle} for this function, shown in \ref{fig:djoracle}. The 
oracle in this example is for a balanced function. 
(See also \cite{watrous:lecnotes} or
\cite{neilsen2000:QuantumComputationAndInfo}).

The function \inlqpl{dj} creates an input list for the function, 
applies the Hadamard transform to all the elements of that list
and  then applies the oracle. When that is completed, the initial segment
of the list is transformed again by Hadamard and then measured. 


\begin{figure}[htbp]
\lstinputlisting[style=linqpl]{examplecode/dj.qpl}
\caption{\lqpl{} code for the Deutsch-Jozsa algorithm}\label{fig:dj}
\end{figure}

The function \inlqpl{prependNZeroqbs} creates a list of \qubits{} when given
a length and the last value. Assuming the parameters passed to the 
function were $3$ and \ket{1}, this would return the list:
\[[\ket{0},\ket{0},\ket{0},\ket{1}]\]
\begin{figure}[htbp]
\lstinputlisting[style=linqpl]{examplecode/prependnzeros.qpl}
\caption{\lqpl{} code to prepend $n$ \ket{0}'s to a \qubit}\label{fig:addnzerps}
\end{figure}

The \inlqpl{initList} function removes the last element of a list.
\begin{figure}[htbp]
\lstinputlisting[style=linqpl]{examplecode/initList.qpl}
\caption{\lqpl{} code accessing initial part of list}\label{fig:initlist}
\end{figure}

The \inlqpl{measureInputs} function recursively measures the 
\qubits{} in a list. If any of them measure to $1$, it returns the 
value \inlqpl{Balanced}. If all of them measure to $0$, it returns
the value \inlqpl{Constant}.

\begin{figure}[htbp]
\lstinputlisting[style=linqpl]{examplecode/measureInps.qpl}
\caption{\lqpl{} code to measure a list of  \qubits}\label{fig:measureinps}
\end{figure}

The oracle for the Deutsch-Jozsa algorithm is normally assumed to be
a given. An actual implementation such as this requires an actual function
to be provided. The function \inlqpl{djoracle} effects a transformation on
the \qubits{} in the input list by delegating to the function 
\inlqpl{bal}.

\begin{figure}[htbp]
\lstinputlisting[style=linqpl]{examplecode/djoracle.qpl}
\caption{\lqpl{} code for Deutsch-Jozsa oracle}\label{fig:djoracle}
\end{figure}
